% -----------------------------------------------------------------------
% Template Skripsi untuk MIPA
% 
% @author Yusuf Syaifudin
% @created 29/02/2016
% 
% -----------------------------------------------------------------------

\documentclass[ugmskripsi]{ugmskripsi}

% ------------------------------------------------------------------------------
% Berisi tambahan package dan konfigurasi untuk masing-masing package.
% Ada baiknya, setiap konfigurasi diletakkan tepat dibawah 
% setelah package dilakukan import (usepackage) agar tidak membingungkan.
% Serta disarankan untuk menambah kegunaan package tersebut agar tidak lupa.
% ------------------------------------------------------------------------------

% font tambahan
\usepackage{textcomp}

% digunakan untuk membuat flowchart
\usepackage{tikz}
\usetikzlibrary{shapes, shapes.misc, arrows, fit, positioning}
\tikzstyle{block} = [rectangle, draw, fill=gray!20, text width=4cm, text centered, rounded corners, minimum height=3em]
\tikzstyle{io} = [trapezium, draw, fill=gray!20, text width=1cm, text centered, rounded corners, minimum height=3em]
\tikzstyle{decision} = [diamond, draw, fill=gray!20, text width=3cm, text centered, minimum height=3em]

\usepackage{float}
\usepackage{booktabs}
\usepackage{pbox}
\usepackage{multirow}
\usepackage[normalem]{ulem}
\useunder{\uline}{\ul}{}

% Untuk hyperlink dan otomatis membuat bookmark
\usepackage{hyperref}

% break tanda /, - dan spasi ke baris baru jika sudah tidak muat
\def\UrlBreaks{\do\/\do-\do\ }

% font url dibuat miring dan dg jenis font ttfamily
\renewcommand{\UrlFont}{\small\ttfamily\itshape}

\usepackage{csquotes}
\usepackage{framed}
\usepackage{enumitem}

% untuk input kode baik dari file atau bukan
\usepackage{listings}

% ----------------------------------------------------------------------------
% Contoh dari file
% ----------------------------------------------------------------------------
% \begin{figure}[H]
%   \lstinputlisting[language=python, firstline=38, lastline=59]{code/linkwalker.py}
%   \caption{Mendapatkan daftar tautan berita pada kompas.com}
%   \label{grab daftar berita kompas}
% \end{figure}
% ----------------------------------------------------------------------------
%
% ----------------------------------------------------------------------------
% Contoh
% ----------------------------------------------------------------------------
% \begin{figure}
% 	\begin{lstlisting}[language=sql]
% 		update train_data_statement set data = replace(data, '“', '"');
% 		update train_data_statement set data = replace(data, '”', '"');

% 		update test_data_statement set data = replace(data, '“', '"');
% 		update test_data_statement set data = replace(data, '”', '"');
% 	\end{lstlisting}
% 	\caption{\textit{Query} SQL untuk melakukan perubahan karakter pada data}
% 	\label{kueri SQL untuk melakukan perubahan karakter pada data}
% \end{figure}
% ------------------------------------------------------------------------------

\usepackage{color}
\usepackage{amsmath}
\usepackage{courier}
\usepackage[scaled=.75]{beramono}

%-----------------------------------------------------------------
% Setting syntax hightlighting
%-----------------------------------------------------------------
\lstset{frame=tb,
  language=Python,
  aboveskip=2mm,
  belowskip=1mm,
  showstringspaces=false,
  columns=flexible,
  basicstyle  = \fontfamily{pcr}\fontsize{8pt}{8pt}\selectfont,
  numbersep=8pt,
  numbers=left,
  numberstyle=\tiny\color{gray},
  keywordstyle=\color{blue},
  commentstyle=\color{dkgreen},
  stringstyle=\color{mauve},
  breaklines=true,
  breakatwhitespace=true,
  tabsize=4
}

% Untuk menghilangkan titik-titik pada daftar isi

\usepackage[titles]{tocloft}
\renewcommand{\cftdot}{}

% Untuk membuat multi kolom
\usepackage{etoolbox,refcount}
\usepackage{multicol}

% Konfigurasi multi kolom
% bikin multi kolom
\newcounter{countitems}
\newcounter{nextitemizecount}
\newcommand{\setupcountitems}{%
  \stepcounter{nextitemizecount}%
  \setcounter{countitems}{0}%
  \preto\item{\stepcounter{countitems}}%
}
\makeatletter
\newcommand{\computecountitems}{%
  \edef\@currentlabel{\number\c@countitems}%
  \label{countitems@\number\numexpr\value{nextitemizecount}-1\relax}%
}
\newcommand{\nextitemizecount}{%
  \getrefnumber{countitems@\number\c@nextitemizecount}%
}
\newcommand{\previtemizecount}{%
  \getrefnumber{countitems@\number\numexpr\value{nextitemizecount}-1\relax}%
}
\makeatother
\newenvironment{AutoMultiColItemize}{%
\ifnumcomp{\nextitemizecount}{>}{2}{\begin{multicols}{2}}{}%
\setupcountitems\begin{itemize}}%
{\end{itemize}%
\unskip\computecountitems\ifnumcomp{\previtemizecount}{>}{2}{\end{multicols}}{}}
%end bikin multi kolom

% ------------------------------------------------------------------------------
% Contoh sintaks:
% ------------------------------------------------------------------------------
% \begin{itemize}
%   \item \textit{Reporting verb} yang hadir sebelum entitas pada kutipan langsung:
%   \begin{AutoMultiColItemize}
% 	  \item tutur
% 	  \item kata
% 	  \item ujar
%   \end{AutoMultiColItemize}

%   \item \textit{Reporting verb} yang hadir setelah entitas pada kutipan langsung:
%   \begin{AutoMultiColItemize}
% 	  \item mengatakan
% 	  \item menjawab
%   \end{AutoMultiColItemize}
% \end{itemize}
% ------------------------------------------------------------------------------


% Setting list agar spasi antar list tidak terlalu banyak
\setlist{
listparindent=\parindent,
parsep=0pt
}

% Agar tetap Justify tapi kata tidak dipisah sesuka hati (not hypenation but justified)
\tolerance=1
\emergencystretch=\maxdimen
\hyphenpenalty=10000
\hbadness=10000
\hyphenchar\font=-1
\sloppy

% Agar support longtable
% https://tex.stackexchange.com/questions/639452/create-long-table-in-latex
% need: varwidth, ninecolors
%%%% begin of required preamble
\usepackage{tabularray}
\UseTblrLibrary{varwidth}
\DefTblrTemplate{contfoot-text}{default}{Bersambung di halaman selanjutnya}
\DefTblrTemplate{conthead-text}{default}{(Lanjutan)}

% https://stackoverflow.com/a/16804893
% \usepackage{tablefootnote}
% \usepackage{footnote}
% \makesavenoteenv{tabular}
% \makesavenoteenv{table}

\usepackage{amsmath}   % <-- for \eqref

\usepackage{pgfgantt}



% Konfigurasi variable seperti judul dan lain sebagainya
\titleind{IDENTIFIKASI KALIMAT KUTIPAN DARI TEKS BERITA \textit{ONLINE} BERBAHASA INDONESIA DENGAN METODE BERBASIS ATURAN}

\titleeng{QUOTATIONS IDENTIFICATION FROM INDONESIAN ONLINE NEWS USING RULE-BASED METHOD}

\fullname{Gusti Agung Rama Ayudhya}

\idnum{20/459266/PA/19927}

\examdate{16 Februari 2016}

\degree{Sarjana Komputer}

\yearsubmit{2016}

\program{Ilmu Komputer}

\gelar{Sarjana Komputer}

\headprogram{Azhari SN, Dr., MT}

\dept{Ilmu Komputer dan Elektronika}

\firstsupervisor{Arif Nurwidyantoro, S.Kom., M.Cs}

\firstexaminer{Mhd. Reza M.I Pulungan, M.Sc., Dr.-Ing}

\secondexaminer{Sigit Priyanta, S.Si., M.Kom}



\begin{document}

%-----------------------------------------------------------------
% Disini awal masukan untuk muka skripsi
%-----------------------------------------------------------------

% Cover
\cover

% Halaman judul
\titlepageind 

% Halaman Persetujuan
\approvalpage

% Halaman Pernyataan
\declarepage

% Halaman Persembahan
\acknowledment
\begin{flushright}
\Large\emph\cal{Karya ini ku persembahkan kepada \\
Ibu, Bapak, dan adik-adikku tercinta
serta teman-teman seperjuangan di Ilmu Komputer Universitas Gadjah Mada}
\end{flushright}


%-----------------------------------------------------------------
% Disini akhir masukan untuk muka skripsi
%-----------------------------------------------------------------

% Motto
\motto
\input{BAB_SKRIPSI/BAB0/2_MOTTO}

% Prakata
\preface
Puji syukur penulis panjatkan ke hadapan Ida Hyang Widhi Wasa, Tuhan Yang Maha Esa, karena atas \textit{asung kerta wara nugraha}-Nya, penulis dapat menyelesaikan skripsi ini dengan judul "Optimasi Multiobjektif Penempatan Mesin Virtual dan Penentuan Rute Jaringan pada \textit{Cloud Data Center} Berbasis Algoritma Genetika \textit{Nondominated Sorting}". Skripsi ini disusun sebagai salah satu syarat untuk memperoleh gelar Sarjana Ilmu Komputer pada Program Studi Ilmu Komputer, Fakultas Matematika dan Ilmu Pengetahuan Alam, Universitas Gadjah Mada.

Penulis menyadari bahwa penyusunan skripsi ini tidak terlepas dari dukungan, bimbingan, dan bantuan dari berbagai pihak. Oleh karena itu, pada kesempatan ini, penulis ingin mengucapkan terima kasih yang sebesar-besarnya kepada:

\begin{enumerate}
  \item Ayah dan Ibu, yang telah membimbing, mendukung, mendoakan, serta membiayai penulis hingga dapat menempuh pendidikan di Universitas Gadjah Mada.
  \item Gusti Ayu Bulan Adhistanaya dan Gusti Agung Deva Maheswara, kedua adik penulis, serta seluruh keluarga besar yang senantiasa memberikan semangat dan dukungan selama proses penyusunan skripsi ini.
  \item Bapak Drs. Medi, M.Kom., selaku Dosen Pembimbing, yang telah dengan sabar membimbing, memberikan ilmu, arahan, masukan, serta koreksi selama proses penulisan skripsi ini.
  \item Almarhumah Ibu Anny Kartika Sari, S.Si., M.Sc. dan Bapak Lukman Heryawan, S.T., M.T., selaku Dosen Pembimbing Akademik, atas segala bimbingan, nasihat, dan bantuan selama penulis menempuh studi di Program Studi Ilmu Komputer.
  \item Tim Penguji, yang telah memberikan kritik, saran, serta masukan berharga untuk penyempurnaan skripsi ini.
  \item Seluruh dosen dan staf Fakultas MIPA UGM, khususnya di Program Studi Ilmu Komputer, yang telah memberikan ilmu dan dukungan selama proses studi.
  \item Teman-teman Ilmu Komputer Angkatan 2020 dan OmahTI, yang telah menjadi rekan seperjuangan dan selalu siap membantu, baik dalam perkuliahan maupun dalam penyusunan skripsi.
  \item Rekan-rekan Tim KKN Punung Periode 4 Tahun 2023 Unit JI-081 (Senandung Punung), dan warga Desa Wareng, Kabupaten Pacitan, Jawa Timur, atas kebersamaan, pengalaman, dan kisah tak terlupakan selama lima puluh hari pengabdian.
  \item Seluruh staf Perpustakaan dan Arsip Universitas Gadjah Mada, yang telah menyediakan fasilitas dan referensi penting dalam mendukung penyusunan skripsi ini.
  \item Seluruh pihak yang tidak dapat disebutkan satu per satu, yang turut membantu dan mendukung penulis selama proses penulisan skripsi ini.
\end{enumerate}

Penulis menyadari bahwa skripsi ini masih jauh dari sempurna. Oleh karena itu, penulis terbuka atas segala kritik dan saran yang membangun demi perbaikan ke depan. Penulis berharap skripsi ini dapat memberikan manfaat, baik dalam pengembangan ilmu komputer maupun sebagai referensi bagi penelitian selanjutnya.

\vspace{1.5cm}
\begin{tabular}{p{7.5cm}c}
&Yogkarta, Mei 2025\\
&\\
&\\
&\space Penulis
\end{tabular}
\vfill


%-----------------------------------------------------------------
% Daftar Isi
%-----------------------------------------------------------------
\newpage
\phantomsection
\addcontentsline{toc}{chapter}{\contentsname}
\tableofcontents
%-----------------------------------------------------------------
% Akhir Daftar Isi
%-----------------------------------------------------------------

%-----------------------------------------------------------------
% Daftar Tabel
%-----------------------------------------------------------------
\newpage
\phantomsection
\addcontentsline{toc}{chapter}{\listtablename}
\listoftables
%-----------------------------------------------------------------
% Akhir Daftar Tabel
%-----------------------------------------------------------------

%-----------------------------------------------------------------
% Daftar Gambar
%-----------------------------------------------------------------
\newpage
\phantomsection
\addcontentsline{toc}{chapter}{\listfigurename}
\listoffigures
%-----------------------------------------------------------------
% Akhir Daftar Gambar
%-----------------------------------------------------------------


%-----------------------------------------------------------------
%Disini awal masukan Intisari
%-----------------------------------------------------------------
\begin{abstractind}
	Komputasi awan telah menjadi infrastruktur utama dalam teknologi informasi karena kemampuannya menyediakan sumber daya secara \textit{scalable} dan elastis sesuai permintaan pengguna. Kemampuan ini didukung oleh virtualisasi, yang memungkinkan penyewaan sumber daya tanpa pengelolaan langsung oleh pengguna, sekaligus meningkatkan efisiensi dan keberlanjutan \textit{data center}. Salah satu tantangan utama dalam komputasi awan adalah penempatan VM (\textit{virtual machine} atau mesin virtual) pada PM (\textit{physical machine} atau mesin fisik) dan rekayasa lalu lintas (\textit{traffic engineering}) jaringan \textit{data center}, yang harus mempertimbangkan berbagai efisiensi operasional, seperti konsumsi energi dan efisiensi sumber daya, serta \textit{Quality of Service} (QoS), seperti alokasi \textit{bandwidth} dan \textit{latency} komunikasi antar-VM.

Optimasi penempatan VM dan penentuan rute menjadi semakin kompleks karena adanya kendala yang sering kali saling bertentangan, sehingga lebih cocok diselesaikan menggunakan metode optimasi multiobjektif. Masalah ini bersifat \textit{NP-complete}, sehingga lebih cocok diselesaikan menggunakan algoritma heuristik dibandingkan algoritma eksak. Penelitian ini menggunakan NSGA-III (\textit{Nondominated Sorting Genetic Algorithm} III), sebuah algoritma genetika untuk menyelesaikan masalah multiobjektif, dengan tujuan mengoptimalkan konsumsi energi, efisiensi sumber daya dalam penempatan VM, alokasi \textit{bandwidth}, dan \textit{latency} komunikasi dalam penentuan rute jaringan secara bersamaan.

Evaluasi performa NSGA-III dilakukan melalui simulasi menggunakan CloudSim Plus, dengan mempertimbangkan berbagai skenario seperti topologi jaringan, kebutuhan sumber daya VM, kapasitas PM, serta parameter NSGA-III. Performa NSGA-III akan diukur menggunakan beberapa metrik, termasuk \textit{hypervolume} dan rata-rata jarak antargenerasi. Selain itu, solusi yang diperoleh NSGA-III akan dibandingkan dengan solusi eksak yang diperoleh \textit{LP solver} (pemecah pemrograman linier) serta beberapa kombinasi algoritma heuristik untuk penempatan VM dan penentuan rute.

\\
\textbf{Kata Kunci: }Komputasi awan, \textit{data center}, penempatan mesin virtual, konsumsi energi, efisiensi sumber daya, masalah \textit{bin packing}, perutean jaringan, alokasi \textit{bandwidth}, \textit{latency}, masalah \textit{multicommodity flow}, metode optimasi multiobjektif, \textit{Nondominated Sorting Genetic Algorithm III} (NSGA-III) 

\end{abstractind}
%-----------------------------------------------------------------
%Disini akhir masukan Intisari
%-----------------------------------------------------------------

%-----------------------------------------------------------------
%Disini awal masukan untuk Abstract
%-----------------------------------------------------------------
\begin{abstracteng}
  Cloud computing has become a key infrastructure in information technology due to its ability to provide scalable and elastic resources on demand. This capability is supported by virtualization, which enables resource rental without direct management by users while also improving efficiency and sustainability of data centers. One of the main challenges in cloud computing is the placement of virtual machines (VMs) on physical machines (PMs) and traffic engineering in data center networks, which must consider various operational efficiency factors, such as energy consumption and resource utilization, as well as Quality of Service (QoS) aspects, such as bandwidth allocation and communication latency between VMs. 

VM placement and network routing optimization becomes increasingly complex due to conflicting constraints, making multiobjective optimization methods suitable for solving this problem. On the top of that, this problem is NP-complete, making heuristic algorithms more suitable than exact algorithms. This study employs NSGA-III (Nondominated Sorting Genetic Algorithm III), a genetic algorithm for solving multi-objective problems, aiming to optimize energy consumption, resource efficiency in VM placement, bandwidth allocation, and communication latency in network routing simultaneously.

The performance of NSGA-III is evaluated through simulations using CloudSim Plus, considering various scenarios such as network topology, VM resource demands, PM capacity, and NSGA-III parameters. The performance of NSGA-III is measured using several metrics, including hypervolume and the average intergenerational distance. Additionally, the solutions obtained by NSGA-III are compared with exact solutions obtained from linear programming (LP) solvers and several heuristic algorithm combinations for VM placement and routing.

\\
\textbf{Keywords: }cloud computing, data center, virtual machine placement, energy consumption, resource wastage, bin packing problem, network routing, bandwidth allocation, latency, multicommodity flow problem, multiobjective optimization method, Nondominated Sorting Genetic Algorithm

\end{abstracteng}
%-----------------------------------------------------------------
%Disini akhir masukan Abstract
%-----------------------------------------------------------------


%-----------------------------------------------------------------
% Awal BAB 1
%-----------------------------------------------------------------
\chapter{PENDAHULUAN}
\label{PENDAHULUAN}

	\section{Latar Belakang}
	\label{pendahuluan latar belakang}
	Seiring perkembangan teknologi informasi, \textit{Cloud Computing} menjadi infrastruktur utama bagi penyediaan layanan komputasi. Teknologi ini memungkinkan penggunaan sumber daya seperti penyimpanan, jaringan, dan daya komputasi secara efisien melalui virtualisasi, di mana berbagai aplikasi berjalan pada \textit{Virtual Machine} (VM) yang di-\textit{host} oleh \textit{Virtual Machine Host} (VMH). Dalam lingkungan \textit{cloud}, \texit{load balancing} berperan penting untuk menjaga performa dan kualitas layanan dengan mendistribusikan beban secara merata di seluruh VMH.

Salah satu tantangan utama dalam \textit{cloud computing} adalah memastikan bahwa setiap VMH tidak mengalami \textit{overload} maupun \textit{underutilization}. Salah satu solusi yang banyak digunakan menangani ketidakseimbangan beban adalah migrasi VM, di mana VM dipindahkan dari VMH yang kelebihan beban ke VMH lain dengan beban lebih sedikit. Namun, migrasi VM  harus dioptimalkan agar tidak menimbulkan \textit{overhead} yang tinggi dan gangguan performa.

Pada penelitian sebelumnya, kombinasi \textit{Gene Expression Programming} (GEP) dan \textit{Genetic Algorithm} (GA) telah digunakan untuk memprediksi beban dan mengoptimalkan migrasi VM.\cite{1}. Meskipun GA efektif dalam menentukan solusi optimal, algoritma ini memiliki biaya komputasi yang tinggi, terutama karena setiap iterasi melibatkan proses eksplorasi ruang solusi yang luas dan operator genetika seperti \textit{crossover}, mutasi, dan seleksi. Selain itu, kebutuhan akan populasi besar dan banyaknya generasi yang diperlukan untuk konvergensi menambah waktu eksekusi, sehingga menjadi kurang efisien saat diterapkan pada lingkungan \textit{cloud} dengan skala besar dan dinamika tinggi. 

Oleh karena itu, penelitian ini mengusulkan penggunaan \textit{Ant Colony Optimization} (ACO) sebagai alternatif GA untuk meningkatkan kinerja dan efisiensi dalam \textit{load balancing}. ACO dikenal unggul dalam menyelesaikan masalah optimasi dengan banyak kemungkinan solusi, seperti \textit{job scheduling} dan \textit{routing}, sehingga relevan untuk diterapkan dalam migrasi VM.


	\section{Rumusan Masalah}
	\label{pendahuluan rumusan masalah}
	Berdasarkan latar belakang di atas, rumusan masalah yang akan dibahas dalam penelitian ini adalah:

    Bagaimana performa beberapa metode load balancing pada router Mikrotik ketika diterapkan pada VPS?
    Algoritma load balancing mana yang paling optimal dalam mendistribusikan beban pada jaringan cloud kampus?
    Apa saja tantangan yang muncul dalam implementasi metode load balancing di lingkungan cloud berbasis Mikrotik?


	\section{Batasan Masalah}
	\label{pendahuluan batasan masalah}
	Penelitian ini akan dibatasi pada:

    Penggunaan Mikrotik CHR (Cloud Hosted Router) yang di-deploy pada VPS.
    Pengujian beberapa metode load balancing seperti GEP dan GA serta RIAL.
    Analisis performa akan difokuskan pada latensi, throughput, dan penurunan kinerja selama migrasi VM.


	\section{Tujuan Penelitian}
	\label{pendahuluan tujuan penelitian}
	Penelitian ini bertujuan

\begin{enumerate}
  \item Mengembangkan dan menguji kombinasi GP-ACO sebagai algoritma \textit{load balancing} untuk \textit{cloud computing}. 
  \item Membandingkan performa kombinasi GP-ACO dengan metode GEP-GA dari penelitian sebelumnya. 
  \item Mengidentifikasi tantangan dan solusi dalam penerapan GP-ACO di lingkungan \textit{cloud}.
\end{enumerate}


	\section{Manfaat Penelitian}
	\label{pendahuluan manfaat penelitian}
	Manfaat dari penilitian ini adalah:

\begin{enumerate}
  \item Bagi Peneliti: Memperluas wawasan dalam penerapan algoritma metaheuristik pada \textit{cloud computing}. 
  \item Bagi Kampus: Memberikan solusi praktis dalam optimasi \textit{load balancing} di jaringan kampus. 
  \item Bagi Pengembang Sistem: Menyediakan referensi bagi pengembangan algoritma \textit{load balancing} berbasis kombinasi GP-ACO.
\end{enumerate}


	\section{Metodologi Penelitian}
	\label{pendahuluan metodologi penelitian}
  Metode penelitian yang digunakan adalah sebagai berikut:

\begin{enumerate}
  \item \textbf{Studi Literatur}: Mengkaji teori terkait \textif{load balancing}. 
  \item \textbf{Pengembangan Algoritma}: Membangun algoritma kombinasi GP-ACO dan mengintegrasikannya dalam lingkungan \textit{cloud}. 
  \item \textbf{Implementasi dan Pengujian}: Menguji performa algoritma GP-ACO dan membandingkannya dengan GEP-GA. 
  \item \textbf{Analisis dan Evaluasi}: Menganalisis hasil eksperimen dan memberikan rekomendasi untuk pengembangan lebih lanjut.
\end{enumerate}
	

	\section{Sistematika Penulisan}
	\label{pendahuluan sistematika penulisan}
	


%-----------------------------------------------------------------
% Akhir BAB 1
%-----------------------------------------------------------------


%-----------------------------------------------------------------
% Awal BAB 2
%-----------------------------------------------------------------
\chapter{TINJAUAN PUSTAKA}
\label{TINJAUAN PUSTAKA}
Seperti yang telah dibahas pada bab sebelumnya, masalah penempatan mesin virtual memiliki kompleksitas \textit{NP-complete}. Oleh karena itu, berbagai algoritma heuristik dan metaheuristik telah dikembangkan untuk menangani masalah ini secara lebih efisien.

Metode-metode tersebut dirancang untuk mengoptimalkan berbagai metrik performa, terutama konsumsi daya server dan pemborosan sumber daya. Meskipun berbagai model konsumsi daya server (Ahmed, Bollen \& Alvarez, 2021) dan model penggunaan sumber daya telah dikembangkan, pada sebagian besar penelitian, konsumsi daya server diukur berdasarkan model yang dikembangkan oleh Beloglazov, Abawajy, dan Buyya (2021), sementara pemborosan sumber daya dihitung menggunakan model dari Gao dkk. (2013). Sejumlah penelitian menyederhanakan pengukuran konsumsi energi \textit{data center} dengan cara menghitung banyaknya server yang aktif (sedang menjalankan mesin virtual).  

\section{Algoritma Penempatan Mesin Virtual}
\subsection{Metode Heuristik}
Beberapa adaptasi algoritma klasik untuk masalah \textit{bin packing} seperti \textit{First Fit} (FF), \textit{First Fit Decreasing} (FFD), \textit{Random Fit} (RF), \textit{Best Fit} (BF), dan \textit{Best Fit Decreasing} (BFD), dapat digunakan untuk menentukan penempatan VM yang optimal (Alharabe, Rakrouki \& Aljohani, 2022). Namun, solusi yang didapat belum cukup optimal. Selain itu, algoritma ini lebih cocok digunakan untuk mengoptimalkan satu objektif saja, seperti konsumsi energi. Untuk memperoleh solusi yang lebih optimal, metode seperti MinPR (Azizi, Zandsalimi \& Li, 2020), GRVMP (\textit{Greedy Randomized Virtual Machine Placement}) (Azizi dkk., 2021), dan CRBFF (\textit{Combinated Random Best First Fit}) (Yousefi \& Babamir, 2024) dikembangkan untuk meminimalkan konsumsi energi sekaligus mengurangi pemborosan sumber daya. Selain itu, algoritma non-\textit{greedy} seperti WPRVMP (\textit{Weighted PageRank-based Virtual Machine Placement}) memanfaatkan algoritma \textit{weighted PageRank} untuk mengurangi jumlah server aktif sambil memaksimalkan pemanfaatan sumber daya server tersebut.

\subsection{Metode Metaheuristik}
Pendekatan metaheuristik, khususnya algoritma evolusioner, banyak digunakan dalam optimasi penempatan mesin virtual. Gao dkk. (2013) mengembangkan VMPACS (\textit{Virtual Machine Placement with Ant Colony System}) berbasis ACO (\textit{Ant Colony Optimization}) untuk meminimalkan konsumsi daya server dan pemborosan sumber daya. Alharabe, Rakrouki, dan Aljohani (2022) memperkenalkan HACOS, yang mengintegrasikan ACO dengan \textit{simulated annealing} untuk mengoptimalkan \textit{network traffic} dan tingkat penggunaan maksimum pada \textit{link} jaringan. Liu dkk. (2018) menciptakan OEMACS untuk mengurangi jumlah server aktif dalam \textit{data center}. Wei dkk. (2019) mengembangkan AP-ACO (\textit{Adaptive Parameter Ant Colony Optimization}), yang parameternya dapat beradaptasi, untuk meminimalkan konsumsi daya dan biaya komunikasi antar-VM.


\subsubsection{ACO (\textit{Ant Colony Optimization})}
Pendekatan metaheuristik, khususnya algoritma evolusioner, banyak digunakan dalam optimasi penempatan mesin virtual. Gao dkk. (2013) mengembangkan VMPACS (\textit{Virtual Machine Placement with Ant Colony System}) berbasis ACO (\textit{Ant Colony Optimization}) untuk meminimalkan konsumsi daya server dan pemborosan sumber daya. Alharabe, Rakrouki, dan Aljohani (2022) memperkenalkan HACOS, yang mengintegrasikan ACO dengan \textit{simulated annealing} untuk mengoptimalkan \textit{network traffic} dan tingkat penggunaan maksimum pada \textit{link} jaringan. Liu dkk. (2018) menciptakan OEMACS untuk mengurangi jumlah server aktif dalam \textit{data center}. Wei dkk. (2019) mengembangkan AP-ACO (\textit{Adaptive Parameter Ant Colony Optimization}), yang parameternya dapat beradaptasi, untuk meminimalkan konsumsi daya dan biaya komunikasi antar-VM.


\subsubsection{Algoritma Genetika}
Algoritma genetika banyak digunakan dalam optimasi penempatan VM, termasuk dengan metode pengkodean yang terinspirasi dari masalah \textit{bin packing}. Metode pengkodean standar merepresentasikan solusi sebagai \textit{array} yang menunjukkan indeks kotak tempat setiap item diletakkan. Namun, Falkenauer (1992) mengusulkan \textit{Grouping Genetic Algorithm} (GGA) untuk mengatasi kelemahan pengkodean standar dengan menambahkan daftar label yang diperbolehkan, sehingga \textit{crossover} dan mutasi tetap mempertahankan struktur partisi.

Wu (2021) menerapkan GGA untuk menempatkan VM pada PM identik guna meminimalkan konsumsi energi. Xu \& Fortes (2010) mengombinasikan GGA dengan logika \textit{fuzzy} untuk meminimalkan pemborosan sumber daya, konsumsi daya, dan suhu tertinggi PM, dengan menggunakan \textit{hash table} sebagai representasi kromosom. Liu dkk. (2014) mengadaptasi GGA dalam \textit{Nondominated Sorting Genetic Algorithm} untuk mengoptimalkan jumlah PM aktif, lalu lintas jaringan, dan keseimbangan penggunaan sumber daya, meskipun tanpa mempertimbangkan topologi jaringan secara eksplisit. Sonklin \& Sonklin (2023) menerapkan GGA untuk penempatan VM berdasarkan tipe yang telah ditentukan penyedia layanan \textit{cloud}.

Tang \& Pan (2014) mengasumsikan topologi jaringan \textit{data center} berbasis hierarki tiga tingkat: \textit{core}, \textit{aggregation}, dan \textit{edge}. Mereka mengklasifikasikan komunikasi antar-VM ke dalam empat kategori berdasarkan lokasi VM, dengan tujuan meminimalkan konsumsi energi jaringan dan PM. Meskipun menggunakan algoritma genetika standar, mereka menerapkan algoritma khusus untuk memperbaiki kromosom yang rusak dan prosedur optimasi lokal guna mengurangi jumlah PM aktif.

\subsubsection{Metode Metaheuristik Lainnya}
Selain ACO dan algorithm genetika, algoritma evolusioner lainnya juga diterapkan. Balaji, Kiran, dan Kumar (2023) menggunakan \textit{firefly algorithm} untuk meminimalkan konsumsi daya, sementara Ghetas (2021) menerapkan \textit{monarch butterfly optimization} dalam MBO-VM untuk mengoptimalkan konsumsi daya dan pemborosan sumber daya. Tripathi, Pathak, dan Vidyarthi (2020) memodifikasi BDA (\textit{Binary Dragonfly Algorithm}) menjadi VMPDA (\textit{Virtual Machine Placement using Dragonfly Algorithm}) untuk mengurangi pemborosan sumber daya. Zhao, Zhou, dan Li (2019) mengembangkan GATA, algoritma hibrida berbasis algoritma genetika dan \textit{tabu search}, untuk meminimalkan konsumsi daya dan meningkatkan \textit{load balance}.


\subsection{Metode \textit{Machine Learning}}
Metode \textit{machine learning}, khususnya \textit{reinforcement learning} (RL), juga banyak digunakan. Caviglione (2021) memanfaatkan \textit{deep reinforcement learning} untuk meminimalkan konsumsi daya server, risiko gangguan perangkat keras, dan interferensi antar-VM. Ghasemi, Haghighat, dan Keshavarzi (2024) mengembangkan dua algoritma untuk menentukan penempatan VM yang bertujuan meminimalkan penggunaan energi, mengurangi pemborosan sumber daya, dan memaksimalkan \textit{load balance}: VMPMFuzzyORL dan MRRL. VMPMFuzzyORL mengintegrasikan \textit{reinforcement learning} (RL) dengan sistem \textit{fuzzy}, sementara MRRL menggabungkan RL dengan algoritma \textit{k-means}. Pada MRRL, algoritma \textit{k-means} digunakan untuk membentuk klaster-klaster VM, sedangkan RL memetakan setiap klaster ke server tertentu. Sebaliknya, pada VMPMFuzzyORL, RL langsung digunakan untuk memetakan masing-masing VM ke server tertentu, dengan \textit{reward} dari setiap aksi ditentukan oleh sistem \textit{fuzzy} yang mengevaluasi ketiga metrik performa tersebut. Qin dkk. (2020) memperkenalkan VMPMORL untuk meminimalkan konsumsi energi dan pemborosan sumber daya. Pada VMPMORL, MDP (\textit{Markov Decision Process}) dimodifikasi menjadi MDP multi-objektif di mana \textit{reward signal} dan \textit{$\widehat{Q}$-value} untuk setiap objektif direpresentasikan sebagai vektor \textit{reward} dan vektor \textit{$\widehat{Q}$-value}. Jarak setiap vektor $\widehat{Q}$-value terhadap titik utopia, titik dengan koordinat ke-$i$ yang berupa nilai terbaik untuk fungsi objektif ke-$i$, dihitung menggunakan metrik Chebyshev. Aksi dengan vektor $\widehat{Q}$-value yang memiliki jarak terkecil dari titik utopia dicari menggunakan algoritma $\epsilon$-\textit{greedy}. 

\section{Metode Optimasi Multiobjektif untuk Masalah Penempatan Mesin Virtual}
Beberapa metode yang disebutkan sebelumnya, seperti VMPMORL (Qin dkk, 2020) dan algoritma buatan Caviglione (2021), merumuskan masalah penempatan mesin virtual sebagai optimasi multiobjektif, di mana solusi diperoleh dengan menyeimbangkan setiap objektif (metrik performa) secara bersamaan untuk menghasilkan sejumlah solusi Pareto. Akan tetapi, sebagian besar penelitian yang telah dibahas belum menerapkan pendekatan ini. Hal tersebut dapat ditunjukkan dari penyerdehanaan yang dilakukan oleh metode-metode ini menjadi objektif tunggal melalui skalarisasi fungsi. Misalnya, untuk $n$ fungsi objektif $f_1, f_2, \dots, f_n$, fungsi yang dihasilkan adalah: $f = a_1f_1 + a_2f_2 + \dots + a_nf_n$.

Agar dapat mengeksplorasi solusi Pareto secara efisien, algoritma MOEA (\textit{Multi-Objective Evolutionary Algorithm}) sering digunakan dalam pendekatan ini. MOEA/D (\textit{Multi-Objective Evolutionary Algorithm based on Decomposition}) digunakan untuk mengoptimalkan konsumsi daya server, pemborosan CPU, dan waktu propagasi (Gopu \& Venkataraman, 2019), sementara NSGA-III (\textit{Non-dominated Sorting Genetic Algorithm}) digunakan untuk meminimalkan konsumsi daya, pemborosan sumber daya, dan \textit{network transmission delay} (Gopu dkk., 2023). Ye, Yin, dan Lin mengembangkan EEKnEA (\textit{Energy-Efficient Knee Point-driven Evolutionary Algorithm}) untuk meminimalkan konsumsi daya, memaksimalkan \textit{load balance}, memaksimalkan rata-rata pemanfaatan sumber daya, dan memaksimalkan rata-rata "\textit{robustness}" server. 

Tao dkk. (2016) mengembangkan BGM-BLA (\textit{Binary Graph Matching-Based Bucket Code Learning Algorithm}) untuk mengubah penempatan VM sehingga jumlah server aktif, komunikasi antar-VM, dan biaya migrasi VM menjadi seminimal mungkin. Sesuai dengan namanya, algoritma ini mengombinasikan algoritma \textit{bucket-code learning} dan \textit{binary graph matching}. BGM-BLA dibagi dalam dua tahap: pembentukan grup-grup VM dan menentukan server yang cocok sebagai tempat baru masing-masing grup tersebut. Algoritma \textit{bucket-code learning} digunakan untuk mencari beberapa kandidat solusi optimal, sedangkan binary graph matching digunakan untuk mengevaluasi dan membandingkan kandidat-kandidat tersebut berdasarkan ketiga objektif tersebut. Kemudian, solusi tersebut dieksplorasi melalui tahap \textit{learning} dan mutasi.


\section{Menyelesaikan Masalah Penempatan Mesin Virtual sekaligus Masalah Penentuan Rute Jaringan}
Sebagian besar metode sebelumnya juga belum memanfaatkan topologi jaringan \textit{data center} sebagai informasi penting dalam menentukan penempatan VM, terutama untuk VM yang berkomunikasi dengan VM lain. Bahkan, metode yang mengoptimalkan metrik kinerja jaringan sering kali hanya memodelkan \textit{data center} sebagai sekumpulan server yang dapat saling berkomunikasi dengan \textit{bandwidth} tetap. Untuk mengatasi kekurangan ini, beberapa metode dikembangkan untuk menentukan tidak hanya penempatan VM tetapi juga rute komunikasi antar-VM. Algoritma HACOS merupakan salah satu contoh metode tersebut (Alharabe, Rakrouki \& Aljohani, 2022). Akan tetapi, HACOS mengasumsikan lingkungan \textit{cloud} yang statis. Oleh karena itu, sejumlah algoritma dirancang untuk lingkungan \textit{cloud} yang dinamis, di mana \textit{data center} melayani banyak \textit{tenant} (pengguna) dengan kebutuhan sumber daya yang beragam. Setiap \textit{tenant} dapat masuk dan keluar dari sistem pada waktu yang berbeda, sehingga algoritma-algoritma tersebut bersifat adaptif dan dijalankan secara berkala selama \textit{data center} aktif.

Jiang dkk. (2012) menggagas sebuah algoritma heuristik yang menggunakan teknik aproksimasi rantai Markov untuk menentukan penempatan VM serta memilih \textit{link} komunikasi yang dapat mengurangi jumlah server aktif dan rata-rata tingkat penggunaan \textit{link}. Tidak seperti algoritma sebelumnya di mana lingkungan \textit{cloud} bersifat statis, algoritma ini dirancang untuk lingkungan \textit{cloud} di mana jumlah \textit{tenant} pada waktu tertentu dimodelkan menggunakan antrean $M/M/\infty$ dan algoritma ini dijalankan setiap kali ada \textit{tenant} yang masuk atau keluar dari sistem. Fang dkk. (2013) mengembangkan pendekatan heuristik untuk menentukan penempatan VM dan rute komunikasi antar-VM untuk meminimalkan konsumsi daya server, biaya migrasi VM, dan \textit{delay} komunikasi dalam jaringan. Algoritma ini dirancang khusus untuk \textit{data center} berbasis OpenFlow dengan topologi \textit{fat tree}. Sementara itu, Luo dkk. (2014) mengusulkan algoritma yang meminimalkan biaya komunikasi jaringan dengan memanfaatkan \textit{minimum tree level} antar-VM berdasarkan topologi jaringan dan penempatan VM di \textit{data center}. Algoritma ini terdiri dari dua tahap: pertama, mengevaluasi apakah total \textit{traffic} pada setiap \textit{switch} melebihi ambang batas yang ditentukan; kedua, memigrasikan VM ke server lain jika ambang batas terlampaui dan menentukan ulang rute komunikasi antar-VM. Algoritma ini dijalankan secara berkala pada interval waktu tertentu.


\begin{table}[h]
\centering
\caption{My caption}
\label{my-label}
\begin{tabular}{|l|l|l|l|}
\hline
Nama  & Kegiatan & Algoritma & Perbedaan dengan peneliti \\ \hline

\pbox{1cm}{
	Yusuf
}
& 
\pbox{4cm}{
	Lorem ipsum dolor sit amet, 
  consectetur adipisicing elit, 
  sed do eiusmodtempor incididunt ut labore et dolore magna aliqua. 
  Ut enim ad minim veniam, 
  quis nostrud exercitation ullamco laboris nisi ut aliquip ex ea commodo consequat.
}
&
\pbox{4cm}{
	Lorem ipsum dolor sit amet, 
  consectetur adipisicing elit, 
  sed do eiusmod tempor incididunt ut labore et dolore magna aliqua. 
  Ut enim ad minim veniam, 
  quis nostrud exercitation ullamco laboris nisi ut aliquip ex ea commodo consequat. 
  Duis aute irure dolor in reprehenderit in voluptate velit esse cillum dolore eu fugiat nulla pariatur.
}
& 
\pbox{3cm}{
	Lorem ipsum dolor sit amet, 
  consectetur adipisicing elit, 
  sed do eiusmod tempor incididunt ut labore et dolore magna aliqua.
}
\\ \hline
\end{tabular}
\end{table}


%-----------------------------------------------------------------
% Akhir BAB 2
%-----------------------------------------------------------------


%-----------------------------------------------------------------
% Awal BAB 3
%-----------------------------------------------------------------
\chapter{DASAR TEORI}
\label{DASAR TEORI}

	\section{Representational State Transfer}
	\label{dasar teori rest}
	Komputasi awan atau \textit{cloud computing} merupakan teknologi yang mampu menyajikan sumber daya teknologi informasi sebagai layanan web yang dapat diakses melalui Internet.
Menurut \textit{National Institute of Standards and Technology} (NIST), terdapat lima karakteristik utama yang menjadi pembeda \textit{cloud computing} dari model komputasi lainnya. 

\begin{enumerate}
  \item \textbf{\textit{{On-Demand Self-Service}}} \textbf{(Layanan Mandiri Sesuai Permintaan)} :
  Pengguna dapat mengakses dan mengatur sumber daya komputasi sesuai kebutuhan tanpa perlu interaksi dengan penyedia layanan \textit{cloud}.
  \item \textbf{textit{Broad Network Access}} \textbf{(Akses Jaringan Luas)}
  Layanan cloud dapat diakses dari mana saja melalui Internet.
  \item \textbf{textit{Resource Pooling}} \textbf{(Penggabungan Sumber Daya)}
  Sumber daya digabungkan dalam satu infrastruktur dan dibagikan kepada banyak pengguna melalui model multi-tenant, di mana sumber daya fisik maupun virtual secara dinamis diberikan sesuai dengan permintaaan pengguna.
  \item \textbf{textit{Rapid Elasticity}} \textbf{(Elastisitas Cepat)}
  Kapabilitas komputasi dapat dengan cepat ditingkatkan atau dikurangi sesuai dengan kebutuhan pengguna.
  \item \textbf{textit{Measured Service}} \textbf{(Layanan Terukur)}
  Penggunaan sumber daya dipantau, diukur, dan ditagihkan sesuai dengan konsumsi sebenarnya, seperti model pay-as-you-go.
\end{enumerate}

Kelima karakteristik tersebut memungkinkan cloud computing memberikan efisiensi, fleksibilitas, dan skalabilitas tinggi bagi pengguna. 

Di dalam lingkungan \textit{cloud}, satu atau lebih VM dengan sistem operasi, aplikasi berjalan, dan spesfikasi yang beragam seperti kebutuhan daya komputasi, memori, kapasitas penyimpanan, dan \textit{bandwidth} jaringan minimal dapat berbagi server yang sama. 
Hal ini dimungkinkan berkat kemampuan penyedia layanan \textit{cloud} dalam memvirtualisasikan sumber daya bagi dan membagikan sumber daya kepada berbagai VM. Kemampuan ini memungkinkan penggunaan server yang lebih sedikit dan lebih efektif. 




	\section{JavaScript Object Notation}
	\label{dasar teori json}

		\subsection{Definisi}
		\label{dasar teori definisi json}
		\input{BAB_SKRIPSI/BAB3/2_1_GEP}

		\subsection{Contoh}
		\label{dasar teori contoh json}
		\input{BAB_SKRIPSI/BAB3/2_2_ACO}

%-----------------------------------------------------------------
% Akhir BAB 3
%-----------------------------------------------------------------


%-----------------------------------------------------------------
% Awal BAB 4
%-----------------------------------------------------------------
\chapter{ANALISIS DAN PERANCANGAN SISTEM}
\label{ANALISIS DAN PERANCANGAN SISTEM}

	\section{Deskripsi Umum Sistem}
	\label{rancangan deskripsi umum sistem}
	\input{BAB_SKRIPSI/BAB4/1_DESKRIPSI_UMUM}

	\section{Analisis Kebutuhan Sistem}
	\label{rancangan analisis kebutuhan sistem}
	\input{BAB_SKRIPSI/BAB4/2_ANALISIS_KEBUTUHAN_SISTEM}

	\section{Pembuatan Sistem}
	\label{rancangan pembuatan sistem}

		\subsection{Pembuatan Sistem Pengenalan Entitas Bernama}
		\label{rancangan pembuatan sistem pengenalan entitas bernama}
		\input{BAB_SKRIPSI/BAB4/3_2_SISTEM_PENGENALAN_ENTITAS_BERNAMA}

		\subsection{Pembuatan Sistem Ekstraksi Kalimat Pernyataan}
		\label{rancangan sistem ekstraksi kalimat pernyataan}
		\input{BAB_SKRIPSI/BAB4/3_3_SISTEM_EKSTRAKSI_KALIMAT_PERNYATAAN}

	\section{Rancangan Antarmuka}
	\label{rancangan antarmuka}

		\subsection{Deskripsi}
		\label{rancangan deskripsi antarmuka}
		\input{BAB_SKRIPSI/BAB4/4_1_DESKRIPSI_RANCANGAN_ANTARMUKA}

		\subsection{\textit{Wireframe}}
	    \label{rancangan wireframe antarmuka}
	    Lorem ipsum odor amet, consectetuer adipiscing elit. Cursus viverra fames inceptos neque imperdiet nostra duis. Dignissim arcu at tempor mattis curae sed nascetur aliquet luctus. Netus arcu venenatis semper suscipit consequat. Phasellus congue sodales blandit ultricies donec dignissim. Dapibus at odio penatibus mauris adipiscing fusce sodales. Quisque nullam massa ullamcorper curae neque vehicula ultricies. Primis bibendum etiam velit viverra arcu etiam sed malesuada ut.

Vulputate ad malesuada elementum et mollis parturient sodales. Netus lectus vitae sit risus netus ipsum congue diam. Faucibus nascetur malesuada risus luctus ridiculus. Suspendisse nec ridiculus accumsan justo parturient metus iaculis. Montes nulla ultricies fringilla nascetur nisi dignissim massa lectus sagittis. Mi tellus orci nullam etiam scelerisque pretium inceptos id feugiat. Lacus luctus natoque placerat cursus faucibus. Luctus porta eget orci nullam magna nostra viverra eget.

Aptent accumsan ac torquent nibh magna tincidunt facilisis facilisi. Libero quis dignissim rhoncus aptent sapien faucibus nostra. Hendrerit volutpat faucibus diam sollicitudin aliquet diam lacus. Hac sed est dictum felis lacus congue at potenti. Metus sollicitudin varius suspendisse consequat scelerisque curae. Luctus porttitor cursus vel neque ipsum egestas. At orci sagittis pulvinar curabitur; ipsum adipiscing nullam diam. Pulvinar euismod interdum aliquam commodo augue aliquam erat. Facilisi dictum imperdiet elit arcu erat dignissim neque. Hac tristique potenti; curabitur fusce aenean leo.

Diam euismod facilisis libero in sem. Ad et justo morbi vel justo primis ipsum cras et? Fermentum lacinia faucibus tristique pharetra fringilla ad. Eu ut integer consequat odio molestie. Nisl lectus ornare erat primis amet laoreet ultricies ligula consequat. Nibh tristique integer iaculis eget phasellus est magna. Fames risus rhoncus turpis sem ad netus massa efficitur.

Fames litora imperdiet accumsan nascetur nam arcu cursus. Odio vel sed platea tempor aptent senectus, consectetur conubia. Leo aenean vitae ultrices quis proin sit. Litora dictum torquent interdum morbi velit adipiscing. Nostra pharetra facilisi iaculis bibendum taciti quisque erat. Justo phasellus sed massa convallis turpis magnis facilisis. Dignissim libero sapien phasellus hendrerit ultricies. Adipiscing faucibus sodales justo hendrerit sagittis imperdiet felis maximus.


%-----------------------------------------------------------------
% Akhir BAB 4
%-----------------------------------------------------------------


%-----------------------------------------------------------------
% Awal BAB 5
%-----------------------------------------------------------------
\chapter{IMPLEMENTASI SISTEM}
\label{IMPLEMENTASI SISTEM}

	\section{Spesifikasi}
	\label{implementasi spesifikasi}
	\input{BAB_SKRIPSI/BAB5/1_SPESIFIKASI}

	\section{Implementasi Sistem Pengenalan Entitas Bernama}
	\label{implementasi sistem ner}
	\input{BAB_SKRIPSI/BAB5/2_1_IMPLEMENTASI_SISTEM_NER}

	\section{Implementasi Sistem Ekstraksi Kalimat Pernyataan}
	\label{implementasi sistem ekstraksi kalimat pernyataan}
	\input{BAB_SKRIPSI/BAB5/2_2_IMPLEMENTASI_SISTEM_EKTRAKSI_KALIMAT_PERNYATAAN}

%-----------------------------------------------------------------
% Akhir BAB 5
%-----------------------------------------------------------------



%-----------------------------------------------------------------
% Awal BAB 6
%-----------------------------------------------------------------
\chapter{PENGUJIAN DAN PEMBAHASAN SISTEM}
\label{PENGUJIAN DAN PEMBAHASAN SISTEM}
\input{BAB_SKRIPSI/BAB6/1_PENDAHULUAN}

	\section{Pengujian Sistem Pengenalan Entitas Bernama}
	\label{pengujian sistem ner}
	\input{BAB_SKRIPSI/BAB6/2_PENGUJIAN_SISTEM_NER}

	\section{Pengujian Sistem Ekstraksi Kalimat Pernyataan}
	\label{pengujian sistem ekstraksi kalimat pernyataan}
	\input{BAB_SKRIPSI/BAB6/3_PENGUJIAN_SISTEM_EKSTRAKSI_KALIMAT_PERNYATAAN}

%-----------------------------------------------------------------
% Akhir BAB 6
%-----------------------------------------------------------------


%-----------------------------------------------------------------
% Awal BAB 7
%-----------------------------------------------------------------
\chapter{PENUTUP}
\label{PENUTUP}

	\section{Kesimpulan}
	\label{penutup kesimpulan}
	\input{BAB_SKRIPSI/BAB7/1_KESIMPULAN}

	\section{Saran}
	\label{penutup saran}
	\input{BAB_SKRIPSI/BAB7/2_SARAN}

%-----------------------------------------------------------------
% Akhir BAB 7
%-----------------------------------------------------------------

%-----------------------------------------------------------------
% Awal Daftar Pustaka
%-----------------------------------------------------------------
\begin{thebibliography}{99}
	\addcontentsline{toc}{chapter}{DAFTAR PUSTAKA}

	\bibitem[Crockford(2006)]{Crockford2006}
	Crockford, Douglas., 2006, \textit{The application/json media type for javascript object notation (json)}.
	
	\bibitem[Ahmed et al.(2021)]{AhmedBollenAlvarez2021}
	Ahmed, K. M. U., Bollen, M. H. J., dan Alvarez, M. (2021). A Review of Data Centers Energy Consumption and Reliability Modeling. \textit{IEEE Access}, 9, 152536 - 152563. \textit{https://doi.org/10.1109/ACCESS.2021.3125092}
	
	\bibitem[Fang et al.(2013)]{Fang2013}
	Fang, S., Kanagavelu, R., Lee, B. S., Foh, C. H., dan Aung, K. M. M. (2013). Power-efficient Virtual Machine Placement and Migration in Data Centers. \textit{2013 IEEE International Conference on Green Computing and Communications and IEEE Internet of Things and IEEE Cyber, Physical and Social Computing}, Beijing, China. \textit{https://10.1109/GreenCom-iThings-CPSCom.2013.246}
	
	\bibitem[Jiang et al.(2012)]{Jiang2012}
	Jiang, J. W., Lan, T., Ha, S., Chen, M., dan Chiang, M. (2012). Joint VM Placement and Routing for Data Center Traffic Engineering. \textit{2012 Proceedings IEEE INFOCOM}, Orlando, Amerika Serikat. \textit{https://doi.org/10.1109/INFCOM.2012.6195719}
	
	\bibitem[Luo et al.(2014)]{Luo2014}
	Luo, G., Qian, Z., Dong, M., Ota, K., dan Lu, S. (2014). Network-Aware Re-Scheduling: Towards Improving Network Performance of Virtual Machines in a Data Center. \textit{Lecture Notes in Computer Science}, 255–269. \textit{https://doi.org/10.1007/978-3-319-11197-1-20}
	
	\bibitem[Zhang and Li(2007)]{ZhangLi2007}
	Zhang, Q. dan Li, H. (2007). MOEA/D: A Multiobjective Evolutionary Algorithm Based on Decomposition. \textit{IEEE Transactions on Evolutionary Computation}, 11(6), 712-731. \textit{https://doi.org/10.1109/TEVC.2007.892759}
	
	\bibitem[Alharabe et al.(2022)]{AlharabeRakroukiAljohani2022}
	Alharabe, N., Rakrouki, M. A., dan Aljohani, A. (2022). An Improved Ant Colony Algorithm for Solving a Virtual Machine Placement Problem in a Cloud Computing Environment. \textit{IEEE Access}, 10, 44869-44880. \textit{https://doi.org/10.1109/ACCESS.2022.3170103}
	
	\bibitem[Azizi et al.(2020)]{AziziZandsalimiLi2020}
	Azizi, S., Zandsalimi, M., dan Li, D. (2020). An energy-efficient algorithm for virtual machine placement optimization in cloud data centers. \textit{Cluster Computing}, 23, 3421-3434. \textit{https://doi.org/10.1007/s10586-020-03096-0}
	
	\bibitem[Azizi et al.(2021)]{Azizi2021}
	Azizi, S., Shojafar, M., Abawajy, J., dan Buyya, R. (2021). GRVMP: A Greedy Randomized Algorithm for Virtual Machine Placement in Cloud Data Centers. \textit{IEEE Systems Journal}, 15(2), 2571-2582. \textit{https://doi.org/10.1109/JSYST.2020.3002721}
	
	\bibitem[Yao et al.(2019)]{YaoShenWang2019}
	Yao, W., Shen, Y., dan Wang, D. (2019). A Weighted PageRank-Based Algorithm for Virtual Machine Placement in Cloud Computing. \textit{IEEE Access}, 7, 176369-176381. \textit{https://doi.org/10.1109/ACCESS.2019.2957772}
	
	\bibitem[Yousefi and Babamir(2024)]{YousefiBabamir2024}
	Yousefi, M. dan Babamir, S. M. (2024). A hybrid energy-aware algorithm for virtual machine placement in cloud computing. \textit{Computing}, 106, 1297-1320. \textit{https://doi.org/10.1007/s00607-024-01280}
	
	\bibitem[Liu et al.(2018)]{Liu2018}
	Liu, X. F., Zhan, Z. H., Deng, J. D., Li, Y., Gu, T., dan Zhang, J. (2018). An Energy Efficient Ant Colony System for Virtual Machine Placement in Cloud Computing. \textit{IEEE Transactions on Evolutionary Computation}, 22(1), 113-128. \textit{https://doi.org/10.1109/TEVC.2016.2623803}
	
	\bibitem[Wei et al.(2019)]{Wei2019}
	Wei, W., Gu, H., Lu, W., Zhou, T., dan Liu, X. (2019). Energy Efficient Virtual Machine Placement With an Improved Ant Colony Optimization Over Data Center Networks. \textit{IEEE Access}, 7, 60617-60625. \textit{https://doi.org/10.1109/ACCESS.2019.2911914} 
	
	\bibitem[Falkenauer and Delchambre(1992)]{FalkenauerDelchambre}
	Falkenauer, E. dan Delchambre, A. (1992). A Genetic Algorithm for Bin Packing and Line Balancing. \textit{Proceeding of the 1992 IEEE International Conference on Robotics and Automation}, Nice, Perancis, Mei 1992, 1186-1192. \textit{https://doi.org/10.1109/ROBOT.1992.220088}
	
	\bibitem[Wu(2021)]{Wu2021}
	Wu, X. (2021). A GA-Based Energy Aware Virtual Machine Placement Algorithm for Cloud Data Centers. \textit{2021 12th International Symposium on Parallel Architectures, Algorithm and Programming}. \textit{https://doi.org/10.1109/PAAP54281.2021.9720442}, diakses 15 Maret 2025
	
	\bibitem[Xu and Fortes(2010)]{XuFortes2010}
	Xu, J. dan Fortes, J. A. B. (2010). Multi-objective Virtual Machine Placement in Virtualized Data Center Environments. \textit{2010 IEEE/ACM International Conference on Green Computing and Communications & 2010 IEEE/ACM International Conference on Cyber, Physical and Social Computing}. \textit{https://doi.org/10.1109/GreenCom-CPSCom.2010.137} diakses 15 Maret 2025
	
	\bibitem[Liu(2014)]{Liu2014}
	Liu, C., Shen, C., Li, S., dan Wang, S. (2014). A New Evolutionary Multi-Objective Algorithm to Virtual Machine Placement in Virtualized Data Center. \textit{2014 IEEE 5th International Conference on Software Engineering and Service Science}, Beijing, China, 27-29 Juni 2014. \textit{https://doi.org/10.1109/ICSESS.2014.6933561,} diakses 15 Maret 2025 
	
	\bibitem[Sonklin and Sonklin(2023)]{SonklinSonklin2023}
	Sonklin, C. dan Sonklin K. (2023). A Multi-Objective Grouping Genetic Algorithm for Server Consolidation in Cloud Data Centers\textit{The 20th International Joint Conference on Computer Science and Software Engineering (JCSSE2023)}, Phitsanulok, Thailand, 28 Juni-1 Juli 2023. \textit{https://doi.org/10.1109/JCSSE58229.2023.10202081}
	
	\bibitem[Tang and Pan(2014)]{TangPan2014}
	Tang, M. dan Pan, S. (2014). A Hybrid Genetic Algorithm for the Energy-Efficient Virtual Machine Placement Problem in Data Centers. \textit{Neural Processing Letters}, 211-221. \textit{https://doi.org/10.1007/s11063-014-9339-8}
	
	\bibitem[Balaji et al.(2023)]{BalajiKiranKumar2023}
	Balaji K., Kiran, P. S., dan Kumar, M. S. (2023). Power aware virtual machine placement in IaaS cloud using discrete firefly algorithm. \textit{Applied Nanoscience}, 13, 2003-2011. \textit{https://doi.org/10.1007/s13204-021-02337-x}

	\bibitem[Ghetas(2021)]{Ghetas2021}
	Ghetas, M. (2021). A multi-objective Monarch Butterfly Algorithm for virtual machine placement in cloud computing. \textit{Neural Computing and Applications}, 33, 11011-11025. \textit{https://doi.org/10.1007/s00521-020-05559-2} 
	
	\bibitem[Tripathi(2020)]{TripathiPathakVidyarthi2020}
	Tripathi, A., Pathak, I., dan Vidyarthi, D. P. (2020). Modified Dragonfly Algorithm for Optimal Virtual Machine Placement in Cloud Computing. \textit{Journal of Network and Systems Management}, 28, 1316-1342. \textit{https://doi.org/10.1007/s10922-020-09538-9}
	
	\bibitem[Zhao et al.(2019)]{ZhaoZhouLi2019}
	Zhao, D., Zhou, J., dan Li, K. (2019). An Energy-Aware Algorithm for Virtual Machine Placement in Cloud Computing. \textit{IEEE Access}, 7, 55659-55668. \textit{https://doi.org/10.1109/ACCESS.2019.2913175}
	
	\bibitem[Caviglione et al.(2021)]{Caviglione2021}
	Caviglione, L., Gaggero, M., Paolucci, M., dan Ronco, R. (2021). Deep reinforcement learning for multi-objective placement of virtual machines in cloud datacenters. \textit{Soft Computing}, 25, 12569-12588. \textit{https://doi.org/10.1007/s00500-020-05462-x} 
	
	\bibitem[Ghasemi et al.(2024)]{Ghasemi2024}
	Ghasemi, A., Haghighat, A. T., dan Keshavarzi, A. (2024). Enhancing virtual machine placement efficiency in cloud datacenters: a hybrid approach using multi-objective reinforcement learning and clustering strategies. \textit{Computing}, 106, 2897-2922. \textit{https://doi.org/10.1007/s00607-024-01311-z}
	
	\bibitem[Ghasemi and Keshavarzi(2024)]{GhasemiKeshavarzi2024}
	Ghasemi, A. dan Keshavarzi, A. (2024). Energy-efficient virtual machine placement in heterogenous cloud data centers: a clustering-enhanced multi-objective, multi-reward reinforcement learning approach. \textit{Cluster Computing}, 27, 14149-14166. \textit{https://doi.org/10.1007/s10586-024-04657-3}
	
	\bibitem[Qin et al.(2020)]{Qin2020}
	Qin, Y., Wang, H., Yi, S., Li, X., dan Zhai, L. (2020). Virtual machine placement based on multi-objective reinforcement learning. \textit{Applied Intelligence}, 50, 2370-2383. \textit{https://doi.org/10.1007/s10489-020-01633-3} 
	
	\bibitem[Gopu and Venkataraman(2019)]{GopuVenkataraman2019}
	Gopu, A., Venkataraman, N. (2019). Optimal VM placement in distributed cloud environment using MOEA/D. \textit{Soft Computing}, 23, 11277–11296. \textit{https://doi.org/10.1007/s00500-018-03686-6}
	
	\bibitem[Gopu et al.(2023)]{Gopu2023}
	Gopu, A., Thirugnanasambandam, K., Rajakumar, Al-Ghamdi, A. S., Alshamrani, S. S., Maharajan K., dan Rashid, M. (2023). Energy-efficient virtual machine placement in distributed cloud using NSGA-III algorithm. \textit{Journal of Cloud Computing}, 12(124). \textit{https://doi.org/10.1186/s13677-023-00501-y} 
	
	\bibitem[Ye et al.(2017)]{YeYinLan2017}
	Ye, X., Yin, Y., dan Lan, L. (2017). Energy-Efficient Many-Objective Virtual Machine Placement Optimization in a Cloud Computing Environment. \textit{IEEE Access}, 5, 16006-16020. \textit{https://doi.org/10.1109/ACCESS.2017.2733723}
	
	\bibitem[Tao et al.(2016)]{Tao2016}
	Tao, F., Li, C., Liao, T. W., dan Laili, Y. (2016). BGM-BLA: A New Algorithm for Dynamic Migration of Virtual Machines in Cloud Computing. \textit{IEEE Transactions on Services Computing}, 9(6), 910-925. \textit{https://doi.org/10.1109/TSC.2015.2416928}

% Bab 3

% Cloud Computing

Mell, P. dan Grance, T. (2011). \textit{The NIST Definition of Cloud Computing}. National Institute of Standards and Technology. \textit{https://doi.org/10.6028/NIST.SP.800-145} 
Hill, R., Hirsch, L., Lake, P., dan Moshiri, S. (2013). \textit{Guide to Cloud Computing}: \textit{Principles and Practice}. London: Springer. \textit{https://doi.org/10.1007/978-1-4471-4603-2}
Cloudflare. (n.d.). What is multitenancy? \textit{Cloudflare Learning Center}. \textit{https://www.cloudflare.com/learning/cloud/what-is-multitenancy}
Huawei Technologies Co., Ltd. (2023). \textit{Cloud Computing Technology}. Singapore: Springer. \textit{https://doi.org/10.1007/978-981-19-3026-3}

Beloglazlov, A., Abawajy, J., dan Buyya, R. (2012). Energy-aware resource allocation heuristics for eficient management of data centers for cloud computing. \textit{Future Generation Computer Systems}, 28(5), 755-768. \textit{https://doi.org/10.1016/j.future.2011.04.017}

% Konsumsi Energi

Jin, C., Bai, X., Yang, C., Mao, W., dan Xu, X. (2020). A review of power consumption models of servers in data centers. \textit{Applied Energy}, 265, 114806. \textit{https://doi.org/10.1016/j.apenergy.2020.114806}

Shehabi, A., Smith, S.J., Hubbard, A., Newkirk, A., Lei, N., Siddik, M.A.B., Holecek, B., Koomey,
J., Masanet, E., dan Sartor, D. (2024). \textit{2024 United States Data Center Energy Usage Report}. Berkeley, California: Lawrence Berkeley National Laboratory

Vasques, T. L., Moura, P., dan de Almeida, A. (2019). A review on energy efficiency and demand response with focus on small and medium data centers. \textit{Energy Efficiency}, 12, 1399-1428. \textit{https://doi.org/10.1007/s12053-018-9753-2}

% Optimasi Multiobjektif

Ehrgott, M. (2005). \textit{Multicriteria Optimization}. Edisi Kedua. New York: Springer 

% Penempatan VM

Korte, B. dan Vygen, J. (2012). \textit{Combinatorial Optimization: Theory and Algorithms}. Edisi Kelima. Springer. \textit{https://doi.org/10.1007/978-3-642-24488-9}

Kao, M. Y. (Eds.). (2008). \textit{Encyclopedia of Algorithms}. Edisi Pertama. New York: Springer. \textit{https://doi.org/10.1007/978-0-387-30162-4}

Floudas, C. A. dan Pardalos, P. M. (Eds.). (2009). \textit{Encyclopedia of Optimization}. Edisi Kedua. New York: Springer. \textit{https://doi.org/10.1007/978-0-387-74759-0}

Fatima, A., Javaid, N., Sultana, T., Hussain, W., Bilal, M., Shabbir, S., Asim, Y., Akbar, M., dan Ilahi, M. (2018). Virtual Machine Placement via Bin Packing in Cloud Data Centers. \textit{Electronics}, 7(12), 389. \textit{https://doi.org/10.3390/electronics7120389} 

% Penentuan rute

Hong, C. Y., Mandal, S., Al-Fares, M., Zhu, M., Alimi, R., Naidu B., K., Bhagat, C., Jain, S., Kaimal, J., Liang, S., Mendelev, K., Padgett, S., Rabe, F., Ray, S., Tewari, M., Tierney, M., Zahn, M., Zolla, J., Ong, J., dan Vahdat, A. (2018). B4 and After: Managing Hierarchy, Partitioning, and Asymmetry for Availability and Scale in Google’s Software-Defined WAN. \textit{SIGCOMM '18: Proceedings of the 2018 Conference of the ACM Special Interest Group on Data Communication}. Budapest, Hungaria, 20-25 Agustus 2018. \textit{https://doi.org/10.1145/3230543.3230545} 

Hong, C. Y., Kandula, S., Mahajan, R., Zhang, M., Gill, V., Nanduri, M., dan Wattenhofer, R. (2013). Achieving High Utilization with Software-Driven WAN. \textit{SIGCOMM '13: Proceedings of the 2018 Conference of the ACM Special Interest Group on Data Communication}, Hong Kong, China, 12-16 Agustus 2013. \textit{https://doi.org/10.1145/2486001.2486012}

Ford, A., Raiciu, C., Handley, M., Barre, S., dan Iyengar, J. (2011). Architectural guidelines for multipath TCP development. \textit{Internet Engineering Task Force, Request For Comments 6182}. Tersedia di: \textit{https://www.rfc-editor.org/rfc/rfc6182.html}

% Algoritma Genetika
Sivanandam, S. N. dan Deepa, S. N. (2008). \textit{Introduction to Genetic Algorithms}. Springer 


% NSGA
Deb, K., Pratap, A., Agarwal, S., dan Meyarivan, T. (2002). A Fast and Elitist Multiobjective Genetic Algorithm: NSGA-II. \textit{IEEE Transactions on Evolutionary Computation}, 6(2), 182-197. \textit{https://doi.org/10.1109/4235.996017}

Deb, K. dan Jain, H. (2014). An Evolutionary Many-Objective Optimization Algorithm Using Reference-Point-Based Nondominated Sorting Approach, Part I: Solving Problems With Box Constraints. \textit{IEEE Transactions on Evolutionary Computation}, 18(4), 577-601 . \textit{https://doi.org/10.1109/TEVC.2013.2281535}

Zhou, Y., Chen, Z., dan Zhang, Y. (2017). Ranking Vectors by Means of the Dominance Degree Matrix. \textit{IEEE Transactions on Evolutionary Computation}. \textit{https://doi.org/10.1109/TEVC.2016.2567648}

% MILP 
You, F., Castro, P. M., dan Grossmann, I. E. (2009). Dinkelbach’s algorithm as an efficient method to solve a class of MINLP models for large-scale cyclic scheduling problems. \textit{Computers and Chemical Engineering}, 33, 1879-1889. \textit{https://doi.org/10.1016/j.compchemeng.2009.05.014}

AIMMS. (2023). \textit{Optimization Modeling}. Tersedia di: \textit{https://documentation.aimms.com/_downloads/AIMMS_modeling.pdf}

% Docplex

IBM Decision Optimization CPLEX Modeling for Python (Docplex) V2.25 documentation. Diakses di: \textit{https://ibmdecisionoptimization.github.io/docplex-doc}

\end{thebibliography}


%-----------------------------------------------------------------
% Akhir Daftar Pustaka
%-----------------------------------------------------------------


%-----------------------------------------------------------------
% Awal lampiran
%-----------------------------------------------------------------
\appendix

\chapter{BERKAS JSON UNTUK MODEL SISTEM PENGENALAN ENTITAS BERNAMA}
\label{BERKAS JSON UNTUK MODEL SISTEM PENGENALAN ENTITAS BERNAMA}
\input{BAB_SKRIPSI/BAB9_LAMPIRAN/1_BERKAS_MODEL_NER}

%-----------------------------------------------------------------
% Akhir lampiran
%-----------------------------------------------------------------

\end{document}
