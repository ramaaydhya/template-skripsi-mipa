Komputasi awan telah menjadi infrastruktur utama dalam teknologi informasi karena kemampuannya menyediakan sumber daya secara \textit{scalable} dan elastis sesuai permintaan pengguna. Kemampuan ini didukung oleh virtualisasi, yang memungkinkan penyewaan sumber daya tanpa pengelolaan langsung oleh pengguna, sekaligus meningkatkan efisiensi dan keberlanjutan \textit{data center}. Salah satu tantangan utama dalam komputasi awan adalah penempatan VM (\textit{virtual machine} atau mesin virtual) pada PM (\textit{physical machine} atau mesin fisik) dan rekayasa lalu lintas (\textit{traffic engineering}) jaringan \textit{data center}, yang harus mempertimbangkan berbagai efisiensi operasional, seperti konsumsi energi dan efisiensi sumber daya, serta \textit{Quality of Service} (QoS), seperti alokasi \textit{bandwidth} dan \textit{latency} komunikasi antar-VM.

Optimasi penempatan VM dan penentuan rute menjadi semakin kompleks karena adanya kendala yang sering kali saling bertentangan, sehingga lebih cocok diselesaikan menggunakan metode optimasi multiobjektif. Masalah ini bersifat \textit{NP-complete}, sehingga lebih cocok diselesaikan menggunakan algoritma heuristik dibandingkan algoritma eksak. Penelitian ini menggunakan NSGA-III (\textit{Nondominated Sorting Genetic Algorithm} III), sebuah algoritma genetika untuk menyelesaikan masalah multiobjektif, dengan tujuan mengoptimalkan konsumsi energi, efisiensi sumber daya dalam penempatan VM, alokasi \textit{bandwidth}, dan \textit{latency} komunikasi dalam penentuan rute jaringan secara bersamaan.

Evaluasi performa NSGA-III dilakukan melalui simulasi menggunakan CloudSim Plus, dengan mempertimbangkan berbagai skenario seperti topologi jaringan, kebutuhan sumber daya VM, kapasitas PM, serta parameter NSGA-III. Performa NSGA-III akan diukur menggunakan beberapa metrik, termasuk \textit{hypervolume} dan rata-rata jarak antargenerasi. Selain itu, solusi yang diperoleh NSGA-III akan dibandingkan dengan solusi eksak yang diperoleh \textit{LP solver} (pemecah pemrograman linier) serta beberapa kombinasi algoritma heuristik untuk penempatan VM dan penentuan rute.

\\
\textbf{Kata Kunci: }Komputasi awan, \textit{data center}, penempatan mesin virtual, konsumsi energi, efisiensi sumber daya, masalah \textit{bin packing}, perutean jaringan, alokasi \textit{bandwidth}, \textit{latency}, masalah \textit{multicommodity flow}, metode optimasi multiobjektif, \textit{Nondominated Sorting Genetic Algorithm III} (NSGA-III) 
