Puji syukur penulis panjatkan ke hadapan Ida Hyang Widhi Wasa, Tuhan Yang Maha Esa, karena atas \textit{asung kerta wara nugraha}-Nya, penulis dapat menyelesaikan skripsi ini dengan judul "Optimasi Multiobjektif Penempatan Mesin Virtual dan Penentuan Rute Jaringan pada \textit{Cloud Data Center} Berbasis Algoritma Genetika \textit{Nondominated Sorting}". Skripsi ini disusun sebagai salah satu syarat untuk memperoleh gelar Sarjana Ilmu Komputer pada Program Studi Ilmu Komputer, Fakultas Matematika dan Ilmu Pengetahuan Alam, Universitas Gadjah Mada.

Penulis menyadari bahwa penyusunan skripsi ini tidak terlepas dari dukungan, bimbingan, dan bantuan dari berbagai pihak. Oleh karena itu, pada kesempatan ini, penulis ingin mengucapkan terima kasih yang sebesar-besarnya kepada:

\begin{enumerate}
  \item Ayah dan Ibu, yang telah membimbing, mendukung, mendoakan, serta membiayai penulis hingga dapat menempuh pendidikan di Universitas Gadjah Mada.
  \item Gusti Ayu Bulan Adhistanaya dan Gusti Agung Deva Maheswara, kedua adik penulis, serta seluruh keluarga besar yang senantiasa memberikan semangat dan dukungan selama proses penyusunan skripsi ini.
  \item Bapak Drs. Medi, M.Kom., selaku Dosen Pembimbing, yang telah dengan sabar membimbing, memberikan ilmu, arahan, masukan, serta koreksi selama proses penulisan skripsi ini.
  \item Almarhumah Ibu Anny Kartika Sari, S.Si., M.Sc. dan Bapak Lukman Heryawan, S.T., M.T., selaku Dosen Pembimbing Akademik, atas segala bimbingan, nasihat, dan bantuan selama penulis menempuh studi di Program Studi Ilmu Komputer.
  \item Tim Penguji, yang telah memberikan kritik, saran, serta masukan berharga untuk penyempurnaan skripsi ini.
  \item Seluruh dosen dan staf Fakultas MIPA UGM, khususnya di Program Studi Ilmu Komputer, yang telah memberikan ilmu dan dukungan selama proses studi.
  \item Teman-teman Ilmu Komputer Angkatan 2020 dan OmahTI, yang telah menjadi rekan seperjuangan dan selalu siap membantu, baik dalam perkuliahan maupun dalam penyusunan skripsi.
  \item Rekan-rekan Tim KKN Punung Periode 4 Tahun 2023 Unit JI-081 (Senandung Punung), dan warga Desa Wareng, Kabupaten Pacitan, Jawa Timur, atas kebersamaan, pengalaman, dan kisah tak terlupakan selama lima puluh hari pengabdian.
  \item Seluruh staf Perpustakaan dan Arsip Universitas Gadjah Mada, yang telah menyediakan fasilitas dan referensi penting dalam mendukung penyusunan skripsi ini.
  \item Seluruh pihak yang tidak dapat disebutkan satu per satu, yang turut membantu dan mendukung penulis selama proses penulisan skripsi ini.
\end{enumerate}

Penulis menyadari bahwa skripsi ini masih jauh dari sempurna. Oleh karena itu, penulis terbuka atas segala kritik dan saran yang membangun demi perbaikan ke depan. Penulis berharap skripsi ini dapat memberikan manfaat, baik dalam pengembangan ilmu komputer maupun sebagai referensi bagi penelitian selanjutnya.

\vspace{1.5cm}
\begin{tabular}{p{7.5cm}c}
&Yogkarta, Mei 2025\\
&\\
&\\
&\space Penulis
\end{tabular}
\vfill
