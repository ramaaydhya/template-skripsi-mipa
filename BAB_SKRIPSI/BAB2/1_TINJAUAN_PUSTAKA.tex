Seperti yang telah dibahas pada bab sebelumnya, masalah penempatan mesin virtual memiliki kompleksitas \textit{NP-complete}. Oleh karena itu, berbagai algoritma heuristik dan metaheuristik telah dikembangkan untuk menangani masalah ini secara lebih efisien.

Metode-metode tersebut dirancang untuk mengoptimalkan berbagai metrik performa, terutama konsumsi daya server dan pemborosan sumber daya. Meskipun berbagai model konsumsi daya server (Ahmed, Bollen \& Alvarez, 2021) dan model penggunaan sumber daya telah dikembangkan, pada sebagian besar penelitian, konsumsi daya server diukur berdasarkan model yang dikembangkan oleh Beloglazov, Abawajy, dan Buyya (2021), sementara pemborosan sumber daya dihitung menggunakan model dari Gao dkk. (2013). Sejumlah penelitian menyederhanakan pengukuran konsumsi energi \textit{data center} dengan cara menghitung banyaknya server yang aktif (sedang menjalankan mesin virtual).  

\section{Algoritma Penempatan Mesin Virtual}
\subsection{Metode Heuristik}
Beberapa adaptasi algoritma klasik untuk masalah \textit{bin packing} seperti \textit{First Fit} (FF), \textit{First Fit Decreasing} (FFD), \textit{Random Fit} (RF), \textit{Best Fit} (BF), dan \textit{Best Fit Decreasing} (BFD), dapat digunakan untuk menentukan penempatan VM yang optimal (Alharabe, Rakrouki \& Aljohani, 2022). Namun, solusi yang didapat belum cukup optimal. Selain itu, algoritma ini lebih cocok digunakan untuk mengoptimalkan satu objektif saja, seperti konsumsi energi. Untuk memperoleh solusi yang lebih optimal, metode seperti MinPR (Azizi, Zandsalimi \& Li, 2020), GRVMP (\textit{Greedy Randomized Virtual Machine Placement}) (Azizi dkk., 2021), dan CRBFF (\textit{Combinated Random Best First Fit}) (Yousefi \& Babamir, 2024) dikembangkan untuk meminimalkan konsumsi energi sekaligus mengurangi pemborosan sumber daya. Selain itu, algoritma non-\textit{greedy} seperti WPRVMP (\textit{Weighted PageRank-based Virtual Machine Placement}) memanfaatkan algoritma \textit{weighted PageRank} untuk mengurangi jumlah server aktif sambil memaksimalkan pemanfaatan sumber daya server tersebut.

\subsection{Metode Metaheuristik}
Pendekatan metaheuristik, khususnya algoritma evolusioner, banyak digunakan dalam optimasi penempatan mesin virtual. Gao dkk. (2013) mengembangkan VMPACS (\textit{Virtual Machine Placement with Ant Colony System}) berbasis ACO (\textit{Ant Colony Optimization}) untuk meminimalkan konsumsi daya server dan pemborosan sumber daya. Alharabe, Rakrouki, dan Aljohani (2022) memperkenalkan HACOS, yang mengintegrasikan ACO dengan \textit{simulated annealing} untuk mengoptimalkan \textit{network traffic} dan tingkat penggunaan maksimum pada \textit{link} jaringan. Liu dkk. (2018) menciptakan OEMACS untuk mengurangi jumlah server aktif dalam \textit{data center}. Wei dkk. (2019) mengembangkan AP-ACO (\textit{Adaptive Parameter Ant Colony Optimization}), yang parameternya dapat beradaptasi, untuk meminimalkan konsumsi daya dan biaya komunikasi antar-VM.


\subsubsection{ACO (\textit{Ant Colony Optimization})}
Pendekatan metaheuristik, khususnya algoritma evolusioner, banyak digunakan dalam optimasi penempatan mesin virtual. Gao dkk. (2013) mengembangkan VMPACS (\textit{Virtual Machine Placement with Ant Colony System}) berbasis ACO (\textit{Ant Colony Optimization}) untuk meminimalkan konsumsi daya server dan pemborosan sumber daya. Alharabe, Rakrouki, dan Aljohani (2022) memperkenalkan HACOS, yang mengintegrasikan ACO dengan \textit{simulated annealing} untuk mengoptimalkan \textit{network traffic} dan tingkat penggunaan maksimum pada \textit{link} jaringan. Liu dkk. (2018) menciptakan OEMACS untuk mengurangi jumlah server aktif dalam \textit{data center}. Wei dkk. (2019) mengembangkan AP-ACO (\textit{Adaptive Parameter Ant Colony Optimization}), yang parameternya dapat beradaptasi, untuk meminimalkan konsumsi daya dan biaya komunikasi antar-VM.


\subsubsection{Algoritma Genetika}
Algoritma genetika banyak digunakan dalam optimasi penempatan VM, termasuk dengan metode pengkodean yang terinspirasi dari masalah \textit{bin packing}. Metode pengkodean standar merepresentasikan solusi sebagai \textit{array} yang menunjukkan indeks kotak tempat setiap item diletakkan. Namun, Falkenauer (1992) mengusulkan \textit{Grouping Genetic Algorithm} (GGA) untuk mengatasi kelemahan pengkodean standar dengan menambahkan daftar label yang diperbolehkan, sehingga \textit{crossover} dan mutasi tetap mempertahankan struktur partisi.

Wu (2021) menerapkan GGA untuk menempatkan VM pada PM identik guna meminimalkan konsumsi energi. Xu \& Fortes (2010) mengombinasikan GGA dengan logika \textit{fuzzy} untuk meminimalkan pemborosan sumber daya, konsumsi daya, dan suhu tertinggi PM, dengan menggunakan \textit{hash table} sebagai representasi kromosom. Liu dkk. (2014) mengadaptasi GGA dalam \textit{Nondominated Sorting Genetic Algorithm} untuk mengoptimalkan jumlah PM aktif, lalu lintas jaringan, dan keseimbangan penggunaan sumber daya, meskipun tanpa mempertimbangkan topologi jaringan secara eksplisit. Sonklin \& Sonklin (2023) menerapkan GGA untuk penempatan VM berdasarkan tipe yang telah ditentukan penyedia layanan \textit{cloud}.

Tang \& Pan (2014) mengasumsikan topologi jaringan \textit{data center} berbasis hierarki tiga tingkat: \textit{core}, \textit{aggregation}, dan \textit{edge}. Mereka mengklasifikasikan komunikasi antar-VM ke dalam empat kategori berdasarkan lokasi VM, dengan tujuan meminimalkan konsumsi energi jaringan dan PM. Meskipun menggunakan algoritma genetika standar, mereka menerapkan algoritma khusus untuk memperbaiki kromosom yang rusak dan prosedur optimasi lokal guna mengurangi jumlah PM aktif.

\subsubsection{Metode Metaheuristik Lainnya}
Selain ACO dan algorithm genetika, algoritma evolusioner lainnya juga diterapkan. Balaji, Kiran, dan Kumar (2023) menggunakan \textit{firefly algorithm} untuk meminimalkan konsumsi daya, sementara Ghetas (2021) menerapkan \textit{monarch butterfly optimization} dalam MBO-VM untuk mengoptimalkan konsumsi daya dan pemborosan sumber daya. Tripathi, Pathak, dan Vidyarthi (2020) memodifikasi BDA (\textit{Binary Dragonfly Algorithm}) menjadi VMPDA (\textit{Virtual Machine Placement using Dragonfly Algorithm}) untuk mengurangi pemborosan sumber daya. Zhao, Zhou, dan Li (2019) mengembangkan GATA, algoritma hibrida berbasis algoritma genetika dan \textit{tabu search}, untuk meminimalkan konsumsi daya dan meningkatkan \textit{load balance}.


\subsection{Metode \textit{Machine Learning}}
Metode \textit{machine learning}, khususnya \textit{reinforcement learning} (RL), juga banyak digunakan. Caviglione (2021) memanfaatkan \textit{deep reinforcement learning} untuk meminimalkan konsumsi daya server, risiko gangguan perangkat keras, dan interferensi antar-VM. Ghasemi, Haghighat, dan Keshavarzi (2024) mengembangkan dua algoritma untuk menentukan penempatan VM yang bertujuan meminimalkan penggunaan energi, mengurangi pemborosan sumber daya, dan memaksimalkan \textit{load balance}: VMPMFuzzyORL dan MRRL. VMPMFuzzyORL mengintegrasikan \textit{reinforcement learning} (RL) dengan sistem \textit{fuzzy}, sementara MRRL menggabungkan RL dengan algoritma \textit{k-means}. Pada MRRL, algoritma \textit{k-means} digunakan untuk membentuk klaster-klaster VM, sedangkan RL memetakan setiap klaster ke server tertentu. Sebaliknya, pada VMPMFuzzyORL, RL langsung digunakan untuk memetakan masing-masing VM ke server tertentu, dengan \textit{reward} dari setiap aksi ditentukan oleh sistem \textit{fuzzy} yang mengevaluasi ketiga metrik performa tersebut. Qin dkk. (2020) memperkenalkan VMPMORL untuk meminimalkan konsumsi energi dan pemborosan sumber daya. Pada VMPMORL, MDP (\textit{Markov Decision Process}) dimodifikasi menjadi MDP multi-objektif di mana \textit{reward signal} dan \textit{$\widehat{Q}$-value} untuk setiap objektif direpresentasikan sebagai vektor \textit{reward} dan vektor \textit{$\widehat{Q}$-value}. Jarak setiap vektor $\widehat{Q}$-value terhadap titik utopia, titik dengan koordinat ke-$i$ yang berupa nilai terbaik untuk fungsi objektif ke-$i$, dihitung menggunakan metrik Chebyshev. Aksi dengan vektor $\widehat{Q}$-value yang memiliki jarak terkecil dari titik utopia dicari menggunakan algoritma $\epsilon$-\textit{greedy}. 

\section{Metode Optimasi Multiobjektif untuk Masalah Penempatan Mesin Virtual}
Beberapa metode yang disebutkan sebelumnya, seperti VMPMORL (Qin dkk, 2020) dan algoritma buatan Caviglione (2021), merumuskan masalah penempatan mesin virtual sebagai optimasi multiobjektif, di mana solusi diperoleh dengan menyeimbangkan setiap objektif (metrik performa) secara bersamaan untuk menghasilkan sejumlah solusi Pareto. Akan tetapi, sebagian besar penelitian yang telah dibahas belum menerapkan pendekatan ini. Hal tersebut dapat ditunjukkan dari penyerdehanaan yang dilakukan oleh metode-metode ini menjadi objektif tunggal melalui skalarisasi fungsi. Misalnya, untuk $n$ fungsi objektif $f_1, f_2, \dots, f_n$, fungsi yang dihasilkan adalah: $f = a_1f_1 + a_2f_2 + \dots + a_nf_n$.

Agar dapat mengeksplorasi solusi Pareto secara efisien, algoritma MOEA (\textit{Multi-Objective Evolutionary Algorithm}) sering digunakan dalam pendekatan ini. MOEA/D (\textit{Multi-Objective Evolutionary Algorithm based on Decomposition}) digunakan untuk mengoptimalkan konsumsi daya server, pemborosan CPU, dan waktu propagasi (Gopu \& Venkataraman, 2019), sementara NSGA-III (\textit{Non-dominated Sorting Genetic Algorithm}) digunakan untuk meminimalkan konsumsi daya, pemborosan sumber daya, dan \textit{network transmission delay} (Gopu dkk., 2023). Ye, Yin, dan Lin mengembangkan EEKnEA (\textit{Energy-Efficient Knee Point-driven Evolutionary Algorithm}) untuk meminimalkan konsumsi daya, memaksimalkan \textit{load balance}, memaksimalkan rata-rata pemanfaatan sumber daya, dan memaksimalkan rata-rata "\textit{robustness}" server. 

Tao dkk. (2016) mengembangkan BGM-BLA (\textit{Binary Graph Matching-Based Bucket Code Learning Algorithm}) untuk mengubah penempatan VM sehingga jumlah server aktif, komunikasi antar-VM, dan biaya migrasi VM menjadi seminimal mungkin. Sesuai dengan namanya, algoritma ini mengombinasikan algoritma \textit{bucket-code learning} dan \textit{binary graph matching}. BGM-BLA dibagi dalam dua tahap: pembentukan grup-grup VM dan menentukan server yang cocok sebagai tempat baru masing-masing grup tersebut. Algoritma \textit{bucket-code learning} digunakan untuk mencari beberapa kandidat solusi optimal, sedangkan binary graph matching digunakan untuk mengevaluasi dan membandingkan kandidat-kandidat tersebut berdasarkan ketiga objektif tersebut. Kemudian, solusi tersebut dieksplorasi melalui tahap \textit{learning} dan mutasi.


\section{Menyelesaikan Masalah Penempatan Mesin Virtual sekaligus Masalah Penentuan Rute Jaringan}
Sebagian besar metode sebelumnya juga belum memanfaatkan topologi jaringan \textit{data center} sebagai informasi penting dalam menentukan penempatan VM, terutama untuk VM yang berkomunikasi dengan VM lain. Bahkan, metode yang mengoptimalkan metrik kinerja jaringan sering kali hanya memodelkan \textit{data center} sebagai sekumpulan server yang dapat saling berkomunikasi dengan \textit{bandwidth} tetap. Untuk mengatasi kekurangan ini, beberapa metode dikembangkan untuk menentukan tidak hanya penempatan VM tetapi juga rute komunikasi antar-VM. Algoritma HACOS merupakan salah satu contoh metode tersebut (Alharabe, Rakrouki \& Aljohani, 2022). Akan tetapi, HACOS mengasumsikan lingkungan \textit{cloud} yang statis. Oleh karena itu, sejumlah algoritma dirancang untuk lingkungan \textit{cloud} yang dinamis, di mana \textit{data center} melayani banyak \textit{tenant} (pengguna) dengan kebutuhan sumber daya yang beragam. Setiap \textit{tenant} dapat masuk dan keluar dari sistem pada waktu yang berbeda, sehingga algoritma-algoritma tersebut bersifat adaptif dan dijalankan secara berkala selama \textit{data center} aktif.

Jiang dkk. (2012) menggagas sebuah algoritma heuristik yang menggunakan teknik aproksimasi rantai Markov untuk menentukan penempatan VM serta memilih \textit{link} komunikasi yang dapat mengurangi jumlah server aktif dan rata-rata tingkat penggunaan \textit{link}. Tidak seperti algoritma sebelumnya di mana lingkungan \textit{cloud} bersifat statis, algoritma ini dirancang untuk lingkungan \textit{cloud} di mana jumlah \textit{tenant} pada waktu tertentu dimodelkan menggunakan antrean $M/M/\infty$ dan algoritma ini dijalankan setiap kali ada \textit{tenant} yang masuk atau keluar dari sistem. Fang dkk. (2013) mengembangkan pendekatan heuristik untuk menentukan penempatan VM dan rute komunikasi antar-VM untuk meminimalkan konsumsi daya server, biaya migrasi VM, dan \textit{delay} komunikasi dalam jaringan. Algoritma ini dirancang khusus untuk \textit{data center} berbasis OpenFlow dengan topologi \textit{fat tree}. Sementara itu, Luo dkk. (2014) mengusulkan algoritma yang meminimalkan biaya komunikasi jaringan dengan memanfaatkan \textit{minimum tree level} antar-VM berdasarkan topologi jaringan dan penempatan VM di \textit{data center}. Algoritma ini terdiri dari dua tahap: pertama, mengevaluasi apakah total \textit{traffic} pada setiap \textit{switch} melebihi ambang batas yang ditentukan; kedua, memigrasikan VM ke server lain jika ambang batas terlampaui dan menentukan ulang rute komunikasi antar-VM. Algoritma ini dijalankan secara berkala pada interval waktu tertentu.


\begin{table}[h]
\centering
\caption{My caption}
\label{my-label}
\begin{tabular}{|l|l|l|l|}
\hline
Nama  & Kegiatan & Algoritma & Perbedaan dengan peneliti \\ \hline

\pbox{1cm}{
	Yusuf
}
& 
\pbox{4cm}{
	Lorem ipsum dolor sit amet, 
  consectetur adipisicing elit, 
  sed do eiusmodtempor incididunt ut labore et dolore magna aliqua. 
  Ut enim ad minim veniam, 
  quis nostrud exercitation ullamco laboris nisi ut aliquip ex ea commodo consequat.
}
&
\pbox{4cm}{
	Lorem ipsum dolor sit amet, 
  consectetur adipisicing elit, 
  sed do eiusmod tempor incididunt ut labore et dolore magna aliqua. 
  Ut enim ad minim veniam, 
  quis nostrud exercitation ullamco laboris nisi ut aliquip ex ea commodo consequat. 
  Duis aute irure dolor in reprehenderit in voluptate velit esse cillum dolore eu fugiat nulla pariatur.
}
& 
\pbox{3cm}{
	Lorem ipsum dolor sit amet, 
  consectetur adipisicing elit, 
  sed do eiusmod tempor incididunt ut labore et dolore magna aliqua.
}
\\ \hline
\end{tabular}
\end{table}
