Seiring perkembangan teknologi informasi, \textit{Cloud Computing} menjadi infrastruktur utama bagi penyediaan layanan komputasi. Teknologi ini memungkinkan penggunaan sumber daya seperti penyimpanan, jaringan, dan daya komputasi secara efisien melalui virtualisasi, di mana berbagai aplikasi berjalan pada \textit{Virtual Machine} (VM) yang di-\textit{host} oleh \textit{Virtual Machine Host} (VMH). Dalam lingkungan \textit{cloud}, \texit{load balancing} berperan penting untuk menjaga performa dan kualitas layanan dengan mendistribusikan beban secara merata di seluruh VMH.

Salah satu tantangan utama dalam \textit{cloud computing} adalah memastikan bahwa setiap VMH tidak mengalami \textit{overload} maupun \textit{underutilization}. Salah satu solusi yang banyak digunakan menangani ketidakseimbangan beban adalah migrasi VM, di mana VM dipindahkan dari VMH yang kelebihan beban ke VMH lain dengan beban lebih sedikit. Namun, migrasi VM  harus dioptimalkan agar tidak menimbulkan \textit{overhead} yang tinggi dan gangguan performa.

Pada penelitian sebelumnya, kombinasi \textit{Gene Expression Programming} (GEP) dan \textit{Genetic Algorithm} (GA) telah digunakan untuk memprediksi beban dan mengoptimalkan migrasi VM.\cite{1}. Meskipun GA efektif dalam menentukan solusi optimal, algoritma ini memiliki biaya komputasi yang tinggi, terutama karena setiap iterasi melibatkan proses eksplorasi ruang solusi yang luas dan operator genetika seperti \textit{crossover}, mutasi, dan seleksi. Selain itu, kebutuhan akan populasi besar dan banyaknya generasi yang diperlukan untuk konvergensi menambah waktu eksekusi, sehingga menjadi kurang efisien saat diterapkan pada lingkungan \textit{cloud} dengan skala besar dan dinamika tinggi. 

Oleh karena itu, penelitian ini mengusulkan penggunaan \textit{Ant Colony Optimization} (ACO) sebagai alternatif GA untuk meningkatkan kinerja dan efisiensi dalam \textit{load balancing}. ACO dikenal unggul dalam menyelesaikan masalah optimasi dengan banyak kemungkinan solusi, seperti \textit{job scheduling} dan \textit{routing}, sehingga relevan untuk diterapkan dalam migrasi VM.
