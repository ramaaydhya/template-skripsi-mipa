Perkembangan teknologi informasi telah mendorong adopsi Cloud Computing (Komputasi Awan) yang menawarkan layanan infrastruktur, platform, dan perangkat lunak berbasis internet. Salah satu layanan yang populer adalah Infrastructure as a Service (IaaS), di mana pengguna dapat memanfaatkan sumber daya komputasi, penyimpanan, dan jaringan tanpa harus memiliki perangkat keras secara fisik. Teknologi ini memungkinkan adanya virtualisasi sumber daya seperti Virtual Machine (VM) yang memberikan fleksibilitas dan efisiensi dalam pengelolaan beban komputasi.

Pada cloud computing, pengelolaan beban menjadi tantangan utama, terutama ketika beberapa Virtual Machine Host (VMH) mengalami beban yang tidak seimbang. Beban yang berlebih pada VMH tertentu dapat menurunkan Quality of Service (QoS) dan melanggar Service Level Agreement (SLA) yang telah disepakati antara penyedia layanan cloud dan penggunanya. Untuk menjaga performa sistem, load balancing diterapkan untuk mendistribusikan beban secara merata di seluruh VMH.

Load balancing di cloud computing dapat dilakukan melalui berbagai metode, salah satunya adalah migrasi VM. Proses migrasi ini memindahkan VM dari VMH yang mengalami kelebihan beban ke VMH yang lebih sedikit terbebani. Namun, terdapat tantangan dalam melakukan migrasi yang efisien, seperti meminimalkan penurunan kinerja VM selama migrasi dan mengurangi frekuensi komunikasi antar-VM yang berlebihan.

Beberapa metode telah dikembangkan untuk mengatasi tantangan ini, termasuk Gen Expression Programming (GEP) dan Genetic Algorithm (GA) yang digunakan untuk memprediksi beban masa depan pada VMH dan mengoptimalkan proses migrasi. Metode lain seperti Resource Intensity Aware Load Balancing (RIAL) juga diperkenalkan untuk memberikan bobot dinamis kepada sumber daya berdasarkan intensitas penggunaannya dan memilih VM yang akan dimigrasikan menggunakan pendekatan Multi-Criteria Decision Making (MCDM).

Penelitian ini akan membandingkan beberapa metode load balancing, seperti metode berbasis GEP dan GA serta RIAL, untuk melihat efektivitasnya dalam distribusi beban di lingkungan cloud computing. Studi ini juga akan mengimplementasikan metode-metode tersebut pada router Mikrotik yang di-host di VPS milik kampus untuk menguji performanya dalam kondisi nyata. 
