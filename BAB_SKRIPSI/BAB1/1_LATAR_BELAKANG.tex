Seiring perkembangan teknologi informasi, \textit{cloud computing} (komputasi awan) telah menjadi infrastruktur utama bagi penyedia layanan komputasi. Komputasi awan memungkinkan penggunaan sumber daya—seperti penyimpanan, jaringan, daya komputasi, basis data, \textit{platform}, dan layanan aplikasi—secara efisien melalui Internet. Model komputasi ini semakin diminati karena mampu menyediakan sumber daya yang bersifat \textit{scalable} dan elastis sesuai permintaan pengguna (\textit{on-demand}).

Kemampuan ini dimungkinkan oleh teknologi virtualisasi. Virtualisasi memungkinkan pengguna menyewa dan mengakses sumber daya yang sudah diabstraksi tanpa perlu memiliki, mengelola, atau merawatnya secara langsung. Virtualisasi juga mendukung pendekatan \textit{multi-tenancy}, yaitu penggunaan infrastruktur yang sama oleh banyak pengguna, sehingga penyedia layanan dapat menghemat biaya operasional (Cloudflare, t.t.). Dengan adanya virtualisasi, VM (\textit{virtual machine} atau mesin virtual) dapat dibuat sebagai emulasi perangkat keras fisik yang mampu menjalankan sistem operasi dan aplikasi layaknya komputer fisik. VM ini dapat dengan mudah dijalankan, diskalakan, dimigrasikan, direplikasikan, atau dihancurkan, sehingga meningkatkan skalabilitas, elastisitas, {fault-tolerance}, ketersediaan, penghindaran bencana (\textit{disaster avoidance}), dan pemulihan bencana (\textit{disaster recovery}) (Hill dkk., 2013).

Namun, supaya dapat digunakan, VM harus ditempatkan pada PM (\textit{physical machine} atau mesin fisik). Penempatan VM menjadi tantangan utama bagi penyedia layanan komputasi awan karena harus memperhatikan kebutuhan sumber daya VM serta kapasitas PM. Penempatan yang optimal dapat meningkatkan efisiensi data center dan metrik performa, seperti minimalisasi konsumsi energi, efisiensi penggunaan sumber daya, serta minimalisasi biaya komunikasi antar-VM.

Komunikasi antar-VM dalam \textit{cloud data center} juga memainkan peran penting dalam kinerja aplikasi. \textit{Cloud data center} skala besar, yang biasanya memiliki ribuan PM terhubung oleh perangkat jaringan, sering kali digunakan untuk mendukung aplikasi yang terdiri atas beberapa komponen saling bergantung. Komponen yang saling berkomunikasi melalui jaringan dapat menyebabkan \textit{delay}, yang berdampak pada jumlah tugas (\textit{task}) yang dapat diproses oleh VM per satuan waktu. Penempatan VM dan pemilihan jalur komunikasi yang ideal mampu mengurangi \textit{delay} komunikasi dengan menempatkan VM yang saling berkomunikasi intensif pada PM yang sama atau PM yang terhubung oleh jalur jaringan sesingkat mungkin. 

Selain itu, manajemen energi di data center menjadi isu krusial dalam komputasi awan. Konsumsi energi tidak hanya memengaruhi biaya operasional tetapi juga berdampak pada lingkungan, seperti emisi karbon yang dihasilkan oleh perangkat fisik, terutama server. Menurut survei yang dilakukan oleh Lawrence Berkeley National Laboratory (2024), sekitar 60\% total konsumsi energi di data center berasal dari mesin fisik. Mesin fisik \textit{idle} (menyala tetapi tidak menjalankan \textit{task} apapun) rata-rata mengonsumsi energi sebesar 70\% dari energi yang digunakan oleh mesin fisik dengan utilisasi CPU maksimal (Beloglazov, Abawajy, dan Buyya, 2012). Dengan penempatan VM yang ideal, lebih banyak VM dapat ditempatkan pada PM dengan tingkat utilisasi tinggi. Mesin fisik yang \textit{idle} dapat dimatikan, sehingga jumlah perangkat aktif berkurang dan konsumsi energi \textit{data center} dapat ditekan. Pendekatan ini tidak hanya meningkatkan efisiensi energi tetapi juga mendukung keberlanjutan lingkungan.

%% Penempatan VM yang tidak optimal dapat menyebabkan ketidakseimbangan sumber daya pada mesin fisik dan fragmentasi (Alsbatin, Oz, dan Ulusoy, 2020). Dalam konteks \textit{data center} komputasi awan, fragmentasi adalah kondisi dimana permintaan alokasi suatu sumber daya tidak dapat dipenuhi meskipun total kapasitas yang tersedia bagi melebihi kebutuhan (Gehr dan Schneider, 2009). Hal ini terjadi karena sumber daya tersebut tidak dapat dialokasikan sepenuhnya dari satu mesin fisik saja. Fragmentasi ditandai dengan kapasitas tersisa yang kecil dan tersebar di banyak mesin fisik, sehingga juga menandakan adanya potensi yang terbuang dan menghambat pemenuhan permintaan alokasi secara efisien. 
 %%
Kebutuhan mengoptimalkan beberapa metrik performa sekaligus: konsumsi energi \textit{data center}, efisiensi penggunaan sumber daya komputasi, alokasi \textit{bandwidth}, dan \textit{latency} komunikasi antar-VM, membuat penempatan VM semakin kompleks. Pengoptimalan metrik-metrik tersebut sering kali saling bertentangan, di mana pengoptimalan salah satu metrik dapat memperburuk metrik lainnya. Selain itu, terdapat kendala yang harus dipenuhi, seperti kebutuhan sumber daya dan kapasitas PM, serta pemilihan jalur komunikasi antar-VM, yang membatasi ruang solusi. Untuk itu, setiap metrik yang dipertimbangkan harus dioptimalkan secara setara tanpa memberikan prioritas pada salah satu metrik. Masalah ini sangat cocok untuk diselesaikan menggunakan metode optimasi multiobjektif.

Masalah penempatan VM merupakan salah satu bentuk masalah \textit{bin packing} (Fatima dkk., 2018), sedangkan masalah pemilihan jalur komunikasi antar-VM merupakan salah satu bentuk masalah \textit{multicommodity flow} (Fortz, Gouveia \& Joyce-Moniz, 2017). Keduanya dikenal memiliki kompleksitas \textit{NP-complete}, sehingga secara keseluruhan, masalah ini juga berkompleksitas \textit{NP-complete}. Oleh karena itu, masalah ini kurang cocok diselesaikan dengan algoritma eksak. Sebagai gantinya, berbagai algoritma heuristik dan metaheuristik telah dikembangkan untuk mencari solusi yang cukup optimal dengan pendekatan yang lebih efisien,

Dalam penelitian ini, penulis menggunakan algoritma NSGA-III (\textit{Nondominated Sorting Genetic Algorithm III}), salah satu varian algoritma genetika yang dirancang untuk menyelesaikan masalah multiobjektif. Algoritma ini akan digunakan untuk menempatkan VM pada PM serta memilih rute komunikasi antar-VM, dengan tujuan mengoptimalkan empat metrik metrik performa sekaligus: konsumsi energi \textit{cloud data center}, efisiensi penggunaan sumber daya komputasi, alokasi \textit{bandwidth}, dan \textit{latency} komunikasi antar-VM.

Kinerja NSGA-III akan dievaluasi pada berbagai skenario, mencakup parameter NSGA-III itu sendiri, variasi topologi jaringan, kebutuhan sumber daya setiap VM, kapasitas sumber daya tiap PM, \textit{bandwidth} yang diperlukan oleh pasangan VM yang saling berkomunikasi, serta kapasitas dan \textit{latency} setiap \textit{link} dalam jaringan. Penelitian ini menggunakan CloudSim Plus untuk memodelkan dan mensimulasikan lingkungan komputasi awan beserta algoritma penempatan VM dan pemilihan jalur komunikasinya.
