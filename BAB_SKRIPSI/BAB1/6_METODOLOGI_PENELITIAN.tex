Penelitian ini dilakukan melalui beberapa tahapan sebagai berikut:

\begin{enumerate}
  \item \textbf{Studi Literatur} \\
    Tahap ini mencakup kajian terhadap penelitian terdahulu yang berkaitan dengan penempatan VM, pemilihan rute komunikasi antar-VM, serta metode optimasi multiobjektif berbasis algoritma evolusioner. Literatur yang dikaji mencakup buku, jurnal ilmiah, serta publikasi konferensi yang relevan. Selain itu, studi juga dilakukan terhadap teknologi yang digunakan, seperti NSGA-III sebagai algoritma optimasi, CloudSim Plus untuk simulasi, dan DOCPLEX sebagai \textit{LP solver}.
    
  \item \textbf{Perancangan Algoritma Penempatan VM dan Lingkungan Simulasi} \\
    Pada tahap ini, dilakukan perancangan model optimasi multiobjektif yang mencakup:
    \begin{enumerate}
     \item Formulasi masalah optimasi dalam bentuk fungsi objektif dan kendala.
        
     \item Pemilihan metode representasi kromosom untuk solusi penempatan VM dan pemilihan rute komunikasi.
        
     \item Perancangan operator algoritma genetika untuk NSGA-III, termasuk crossover, mutasi, dan mekanisme perbaikan.
        
     \item Penentuan skenario simulasi, seperti topologi jaringan, konfigurasi VM dan PM, serta parameter simulasi lainnya.
    \end{enumerate}    
  \item \textbf{Implementasi Algoritma} \\
    Tahap ini mencakup implementasi algoritma NSGAIII dalam lingkungan simulasi CloudSim Plus. Implementasi meliputi:
    \begin{enumerate}
     \item Implementasi algoritma NSGA-III ke dalam program Java.
        
     \item Integrasi algoritma NSGA-III dengan CloudSim Plus untuk menilai performa solusi yang dihasilkan.
        
     \item Pengujian awal untuk memastikan algoritma berjalan sesuai dengan perancangan.
    \end{enumerate}        
  \item \textbf{Pengujian Algoritma pada Lingkungan Simulasi dan Evaluasi Kinerja} \\
    Algoritma yang telah diimplementasikan diuji pada berbagai skenario simulasi dengan variasi parameter, seperti:
    \begin{enumerate}
     \item Topologi jaringan yang berbeda (misalnya, FatTree, BCube, DCell).
        
     \item Jumlah VM dan PM yang bervariasi.
    
     \item Kebutuhan sumber daya setiap VM dan kapasitas sumber daya di setiap PM yang bervariasi.
    \end{enumerate}      
    Evaluasi kinerja dilakukan dengan membandingkan kualitas solusi Pareto yang dihasilkan, efisiensi komputasi algoritma, serta stabilitas hasil dalam berbagai skenario.
        
\item \textbf{Penulisan Laporan} \\
    Hasil penelitian yang mencakup formulasi masalah, perancangan algoritma, hasil eksperimen, dan analisis kinerja algoritma didokumentasikan dalam bentuk skripsi.
\end{enumerate}
