Skripsi ini ditulis dalam tujuh bab dengan struktur sebagai berikut:
\begin{enumerate}
 \item[\textbf{Bab I:}] \textbf{Pendahuluan} \\
    Bab ini membahas latar belakang, rumusan masalah, batasan masalah, tujuan penelitian, manfaat penelitian, dan sistematika penulisan skripsi ini.
    
 \item[\textbf{Bab II:}] \textbf{Tinjauan Pustaka} \\
    Bab ini membahas penelitian terdahulu yang berkaitan dengan metode penempatan VM, optimasi multiobjektif, serta pemodelan \textit{cloud data center}. Selain itu, tinjauan terhadap algoritma evolusioner seperti NSGA-III juga disertakan untuk memberikan dasar pengembangan metode dalam penelitian ini.
    
 \item[\textbf{Bab III:}] \textbf{Landasan Teori} \\  
    Bab ini membahas teori dan konsep yang mendukung penelitian, termasuk komputasi awan, metode optimasi multiobjektif, algoritma genetika, dan NSGA-III. Selain itu, bab ini menguraikan cara kerja NSGA-III serta formulasi masalah penempatan VM dan pemilihan jalur komunikasi antar-VM sebagai masalah optimasi multiobjektif.
    
 \item[\textbf{Bab IV:}] \textbf{Analisis dan Rancangan Sistem} \\
    Bab ini membahas analisis masalah, perancangan algoritma, perancangan lingkungan simulasi, serta rancangan evaluasi kinerja algoritma.
    
 \item[\textbf{Bab V:}] \textbf{Implementasi} \\
    Bab ini membahas implementasi algoritma NSGA-III, integrasi NSGA-III dengan lingkungan simulasi CloudSim Plus, pengaturan parameter eksperimen, serta mengevaluasi kinerja algoritma berdasarkan skenario yang dipertimbangkan dalam simulasi.
    
 \item[\textbf{Bab VI:}] \textbf{Hasil dan Pembahasan} \\
    Bab ini membahas hasil evaluasi kinerja algoritma berdasarkan simulasi yang telah dilakukan. Analisis dilakukan terhadap kualitas solusi Pareto, efisiensi komputasi, serta stabilitas algoritma dalam berbagai skenario simulasi.
    
 \item[\textbf{Bab VII:}] \textbf{Kesimpulan} \\
    Bab ini berisi kesimpulan penelitian serta saran untuk pengembangan lebih lanjut.
\end{enumerate}
