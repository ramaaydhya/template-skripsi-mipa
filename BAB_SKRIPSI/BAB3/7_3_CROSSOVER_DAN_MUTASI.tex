Tahap ini tidak dilakukan terhadap $P_0$ sehingga tahap selanjutnya (fast nondominated sorting, pembentukan titik referensi, serta \textit{niching}) dilakukan terhadap $N$ individu saja. Namun, untuk generasi ke-$t$ atau $P_t$, tiap individu akan melakukan \textit{crossover} dengan satu individu lain untuk menghasilkan $N$ buah \textit{offspring}. \textit{Crossover} antarindividu terjadi dengan probabilitas $p_c$. Individu hasil \textit{crossover} oleh anggota $P_t$ diberi notasi $Q_t$. Lalu, setiap anggota $Q_t$ akan bermutasi dengan probabilitas $p_m$. Hasil mutasi $Q_t$ diberi notasi $Q_t'$. Dengan demikian, untuk $t > 0$, terdapat $2N$ individu yang diperoleh pada generasi ke-$t$, yakni $R_t = P_t \cup Q_t'$, sedangkan untuk $t = 0$, $R_t = P_t$.
