\textit{Fast non-dominated sorting} terhadap populasi berukuran $2N$ individu dengan $M$ fungsi objektif membutuhkan $O(MN^2)$ komputasi. Identifikasi titik ideal membutuhkan $O(MN)$ komputasi. Translasi vektor objektif pada $\mathbf{f}(S_t)$ membutuhkan $O(MN)$ komputasi. Identifikasi titik ekstrem membutuhkan $O(M^2N)$ komputasi. Menghitung invers matriks membutuhkan $O(M^3)$ komputasi. Normalisasi tiap vektor pada $\mathbf{f}(S_t)$ menjadi $\bar{\mathbf{f}}(S_t)$ membutuhkan $O(N)$ komputasi. Asosiasi titik referensi dengan vektor ternormalisasi membutuhkan $O(MNH)$ komputasi. Di tahap preservasi \textit{niche}, langkah 2 membutuhkan $O(H)$ komputasi sedangkan langkah 3 dan 5 membutuhkan $O(|F_l|)$ komputasi. Karena perulangan pada tahap preservasi niche dilakukan paling banyak $|F_l|$ kali, total komputasi yang dibutuhkan saat tahap ini adalah $\max\{O(|F_l|^2),O(|F_l| \cdot H)\}$. Pada penelitian ini, $N \approx H$ dan $N > M$. Dengan demikian, kompleksitas waktu algoritma NSGA-III adalah $O(MN^2)$.
