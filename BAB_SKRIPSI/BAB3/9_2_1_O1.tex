Untuk mempermudah pembahasan, fungsi (O1) dapat dinyatakan sebagai berikut:
\begin{equation*}
\displaystyle \sum_{j=1}^{N_P}\sum_{i=1}^{N_V}P_{ij}x_{ij}y_j+\text{PC}_j^\text{idle}
\end{equation*}
di mana
\begin{equation*}
P_{ij}=\displaystyle\frac{v_i^\text{cpu}(\text{PC}_j^\max-\text{PC}_j^\text{idle})}{p_j^\text{cpu}}
\end{equation*}
Karena kebutuhan sumber daya setiap VM dan kapasitas sumber daya di setiap PM diasumsikan selalu tetap, $P_{ij}$ dianggap sebagai konstanta.
Perhatikan bahwa pada ekspresi di atas, terdapat perkalian antara variabel biner $x_{ij}$ dan $y_j$, sehingga ekspresi tersebut tidak bersifat linier. Oleh karena itu, didefinisikan $N_V\cdot N_P$ buah variabel keputusan bantu $w_{ij}=x_{ij}y_j$. Kemudian, beberapa batasan untuk $w_{ij}$ ditambahkan ke dalam model di atas sehingga diperoleh model berikut:
\begin{equation*}
\begin{array}{rlll}
\text{Minimalkan} & \displaystyle \sum_{j=1}^{N_P}\sum_{i=1}^{N_V}P_{ij}w_{ij}+\text{PC}_j^\text{idle} & \text{(O1a)}\\
\text{dengan syarat} & w_{ij} \leq x_{ij} & \text{(W1)}\\
& w_{ij} \leq y_{j}& \text{(W2)}\\
& w_{ij} \geq x_{ij} + y_j - 1& \text{(W3)}\\
\end{array}
\end{equation*}
