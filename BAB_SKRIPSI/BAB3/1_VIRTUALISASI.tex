Komputasi awan atau \textit{cloud computing} merupakan teknologi yang mampu menyajikan sumber daya teknologi informasi sebagai layanan web yang dapat diakses melalui Internet.
Menurut \textit{National Institute of Standards and Technology} (NIST), terdapat lima karakteristik utama yang menjadi pembeda \textit{cloud computing} dari model komputasi lainnya. 

\begin{enumerate}
  \item \textbf{\textit{{On-Demand Self-Service}}} \textbf{(Layanan Mandiri Sesuai Permintaan)} :
  Pengguna dapat mengakses dan mengatur sumber daya komputasi sesuai kebutuhan tanpa perlu interaksi dengan penyedia layanan \textit{cloud}.
  \item \textbf{textit{Broad Network Access}} \textbf{(Akses Jaringan Luas)}
  Layanan cloud dapat diakses dari mana saja melalui Internet.
  \item \textbf{textit{Resource Pooling}} \textbf{(Penggabungan Sumber Daya)}
  Sumber daya digabungkan dalam satu infrastruktur dan dibagikan kepada banyak pengguna melalui model multi-tenant, di mana sumber daya fisik maupun virtual secara dinamis diberikan sesuai dengan permintaaan pengguna.
  \item \textbf{textit{Rapid Elasticity}} \textbf{(Elastisitas Cepat)}
  Kapabilitas komputasi dapat dengan cepat ditingkatkan atau dikurangi sesuai dengan kebutuhan pengguna.
  \item \textbf{textit{Measured Service}} \textbf{(Layanan Terukur)}
  Penggunaan sumber daya dipantau, diukur, dan ditagihkan sesuai dengan konsumsi sebenarnya, seperti model pay-as-you-go.
\end{enumerate}

Kelima karakteristik tersebut memungkinkan cloud computing memberikan efisiensi, fleksibilitas, dan skalabilitas tinggi bagi pengguna. 

Di dalam lingkungan \textit{cloud}, satu atau lebih VM dengan sistem operasi, aplikasi berjalan, dan spesfikasi yang beragam seperti kebutuhan daya komputasi, memori, kapasitas penyimpanan, dan \textit{bandwidth} jaringan minimal dapat berbagi server yang sama. 
Hal ini dimungkinkan berkat kemampuan penyedia layanan \textit{cloud} dalam memvirtualisasikan sumber daya bagi dan membagikan sumber daya kepada berbagai VM. Kemampuan ini memungkinkan penggunaan server yang lebih sedikit dan lebih efektif. 


