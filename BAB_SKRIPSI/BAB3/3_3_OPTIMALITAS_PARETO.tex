
Pada masalah minimalisasi berobjektif tunggal, solusi yang paling optimal adalah solusi dengan nilai objektif paling rendah. Hal ini serupa dengan maksimalisasi berobjektif tunggal. Karena fungsi objektif memiliki nilai dalam himpunan bilangan riil, setiap nilai objektif dapat diurutkan secara linier. Selain itu, nilai maksimum dan minimum yang dapat dicapai oleh fungsi objektif tidak lebih dari satu.

Akan tetapi, membandingkan keoptimalan dua buah solusi pada masalah multiobjektif tidak semudah itu, terlebih lagi dalam menentukan solusi paling optimal. Dalam beberapa kasus, sering kali dijumpai sepasang solusi, $\mathbf{x}^{(1)}$ dan $\mathbf{x}^{(2)}$, di mana solusi $\mathbf{x}^{(1)}$ lebih optimal daripada solusi $\mathbf{x}^{(2)}$ menurut satu atau lebih fungsi objektif, sementara solusi $\mathbf{x}^{(2)}$ lebih optimal daripada solusi $\mathbf{x}^{(1)}$ menurut fungsi objektif lainnya. 

Sebagai contoh, misalkan terdapat sepasang solusi, $\mathbf{x}^{(1)}$ dan $\mathbf{x}^{(2)}$, di mana $\mathbf{f}(\mathbf{x}^{(1)})=(1,4)^T$ dan $\mathbf{f}(\mathbf{x}^{(2)})=(2,3)^T$. Solusi $\mathbf{x}^{(1)}$ tidak bisa dibandingkan terhadap solusi $\mathbf{x}^{(2)}$ karena menurut $f_1$, solusi $\mathbf{x}^{(1)}$ lebih optimal ($1 < 2$), tetapi menurut $f_2$, solusi $\mathbf{x}^{(2)}$ lah yang lebih optimal ($4 > 3$). 

Membandingkan nilai dua buah vektor elemen demi elemen, seperti pada contoh sebelumnya, tidak menghasilkan pengurutan total (\textit{total ordering}), melainkan pengurutan parsial (\textit{partial ordering}) (kecuali vektor berdimensi nol dan satu). Oleh karena itu, mencari vektor nilai objektif "maksimum" atau "minimum" belum tentu dapat dilakukan.  

Selain itu, dalam beberapa masalah multiobjektif, pengoptimalan setiap fungsi objektif tidak selalu dapat dilakukan secara bersamaan. Sering kali ditemukan solusi yang hanya mengoptimalkan sebagian fungsi objektif, tetapi ketika solusi di sekitarnya berusaha mengoptimalkan fungsi yang sebelumnya kurang optimal, nilai fungsi yang semula optimal justru menjadi kurang optimal.

Salah satu strategi klasik untuk memecahkan masalah optimasi multiobjektif adalah mereduksinya menjadi masalah baru dengan objektif tunggal, sehingga dapat diselesaikan dengan metode yang lebih sederhana. Baik masalah asli maupun masalah baru memiliki komponen yang sama, kecuali fungsi objektifnya. Pada masalah baru ini, fungsi objektif dibentuk dengan menggabungkan semua fungsi objektif dari masalah asli menjadi satu fungsi. 

Terdapat tiga pendekatan dalam pembentukan fungsi objektif baru ini: pendekatan penjumlahan berbobot, pendekatan metrik $L_p$ atau fungsi jarak (misalnya jarak Manhattan, jarak Euclid, atau jarak Chebyshev), dan pendekatan \textit{boundary intersection}. Meskipun reduksi ini mampu memperoleh satu solusi optimal Pareto, metode ini masih memiliki kelemahan utama, yakni sensitivitas terhadap bobot yang diberikan kepada tiap fungsi objektif. Pengambil keputusan (\textit{decision maker}) harus menetapkan bobot yang sesuai dengan karakteristik masalah untuk memperoleh solusi optimal. Namun, pemberian bobot juga memerlukan pemahaman tentang skala prioritas tiap fungsi objektif, karena semakin besar bobot yang diberikan kepada suatu fungsi objektif, semakin tinggi prioritas fungsi tersebut sehingga nilai fungsi objektif tersebut mungkin menjadi lebih sensitif terhadap perubahan solusi.

Untuk mengatasi kendala tersebut, sebarang dua solusi dibandingkan berdasarkan optimalitas Pareto. Diberikan dua vektor nilai objektif $\mathbf{z}^{(1)}, \mathbf{z}^{(2)} \in \mathbb{R}^m$, di mana $\mathbf{z}^{(1)} = (z_1^{(1)},z_2^{(1)},\dots,z_m^{(1)})$ dan $\mathbf{z}^{(2)} = (z_1^{(2)},z_2^{(2)},\dots,z_m^{(2)})$, vektor $\mathbf{z}^{(1)}$ disebut mendominasi vektor $\mathbf{z}^{(2)}$ (ditulis $\mathbf{z}^{(1)} \succ \mathbf{z}^{(2)}$) jika berlaku dua kondisi berikut:
\begin{enumerate}
  \item {$z_i^{(1)} \geq z_i^{(2)}$ untuk setiap $i = 1,2,\dots,m$, dan} 
  \item {Terdapat paling tidak satu indeks $j = 1,2,\dots,m$ dengan $z_j^{(1)} > z_j^{(2)}$.} 
\end{enumerate}

Dengan kata lain, $\mathbf{z}^{(1)}$ mendominasi $\mathbf{z}^{(2)}$ apabila untuk setiap fungsi objektif, nilainya pada $\mathbf{z}^{(1)}$ tidak lebih buruk daripada $\mathbf{z}^{(2)}$, dan setidaknya terdapat satu fungsi objektif di mana $\mathbf{z}^{(1)}$ lebih unggul.

Berdasarkan dominasi antar vektor nilai objektif, solusi $\mathbf{x}^{(1)}$ disebut mendominasi solusi $\mathbf{x}^{(2)}$ berdasarkan vektor objektif $\mathbf{f}$  (ditulis $\mathbf{x}^{(1)} \succ_\mathbf{f} \mathbf{x}^{(2)}$) jika $\mathbf{f}(\mathbf{x}^{(1)}) \succ \mathbf{f}(\mathbf{x}^{(2)})$. Sebuah solusi $\mathbf{x}^{*}$ dikatakan optimal Pareto (\textit{Pareto optimal}) terhadap himpunan $S$ jika tidak ada solusi lain dalam $S$ yang mendominasi $\mathbf{x}^*$. Vektor $\mathbf{f}(\mathbf{x}^*)$ disebut sebagai vektor objektif optimal Pareto (\textit{Pareto optimal objective vector}). 

Keistimewaan solusi optimal Pareto adalah bahwa upaya mengoptimalkan satu fungsi objektif akan menyebabkan penurunan performa pada setidaknya satu fungsi objektif lainnya. Himpunan semua solusi optimal Pareto terhadap $S$ disebut himpunan Pareto (\textit{Pareto set}) pada $S$ , sedangkan himpunan semua vektor objektif optimal Pareto-nya disebut sebagai \textit{Pareto front} pada $S$.

Perlu diperhatikan bahwa optimalitas Pareto tidak menghasilkan pengurutan total dan tidak menghilangkan konflik antar fungsi objektif. Meskipun tidak menghasilkan pengurutan total, konsep ini tetap berguna karena dalam banyak masalah multiobjektif, tidak ada solusi tunggal yang optimal secara mutlak di setiap fungsi objektif. Selain itu, konsep ini mengakomodasi keberadaan konflik tersebut dengan menentukan himpunan solusi yang tidak bisa diperbaiki lebih lanjut tanpa mengorbankan aspek yang lain. Metode penentuan ini lebih sistematis daripada hanya membandingkan elemen demi demi tanpa aturan dominasi. Oleh karena itu, \textit{Pareto front} bukanlah solusi akhir, melainkan kumpulan solusi yang harus dianalisis lebih lanjut berdasarkan preferensi tertentu.
