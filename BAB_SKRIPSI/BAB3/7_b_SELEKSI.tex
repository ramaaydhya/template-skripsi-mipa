Di akhir tahap \textit{nondominated sorting}, diperoleh $l$ front. $l$ adalah banyak front minimum yang dibutuhkan untuk mengelompokkan $N$ individu terunggul di antara $R_t$. Definsikan $S_t = \bigcup_{i=1}^l F_i$ sebagai himpunan semua individu yang berada di dalam \textit{front} takterdominasi peringkat $l$ teratas. Maka, $|S_t| \geq N$. Jika $|S_t| = N$, semua anggota $S_t$ menjadi anggota $P_{t+1}$ tanpa melalui tahap seleksi. Artinya, $P_{t+1}=S_t$. Namun, jika $|S_t|>N$, semua individu dari $l-1$ \textit{front} pertama menjadi anggota $P_{t+1}$. Artinya, $S_t \setminus F_l \subset P_{t+1}$. Kemudian, agar $P_{t+1}$ memiliki $N$ anggota, individu dari $F_l$ dipilih menjadi anggota $P_{t+1}$. 

Tahap seleksi dilakukan untuk memilih $N$ individu yang akan menjadi anggota $P_{t+1}$. Karena $R_0$ sudah memiliki $N$ individu, tahap seleksi tidak dilakukan pada generasi ke-0. Akibatnya, seleksi
