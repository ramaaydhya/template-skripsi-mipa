\textit{Cloud data center} memiliki jaringan yang menghubungkan PM dan perangkat jaringan seperti \textit{switch} dan \textit{router} melalui beberapa \textit{link}. Mesin-mesin virtual yang sudah ditempatkan pada PM-nya masing-masing akan saling berkomunikasi dengan kebutuhan \textit{bandwidth} tertentu sebelum proses pemetaan dilakukan. Jika dua VM ditempatkan dalam PM yang sama, komunikasi mereka dapat dilakukan di dalam PM tanpa melalui jaringan. Namun, apabila dua buah VM ditempatkan pada PM yang berbeda, keduanya harus berkomunikasi melewati jaringan dengan membentuk koneksi antara PM tempat mereka dipetakan. Akibatnya, jumlah koneksi serta alokasi \textit{bandwidth} yang diperlukan bergantung pada hasil pemetaan VM. 

Subbab ini membahas notasi serta model optimasi yang digunakan dalam penentuan rute komunikasi dalam jaringan. Dalam model ini, lalu lintas antara setiap pasang PM yang berkomunikasi akan dialokasikan paling banyak ke $k$ buah rute, di mana nilai $k$ ditentukan oleh pengambil keputusan. Dengan demikian, masalah optimasi ini dapat dipandang sebagai masalah \textit{k-splittable multicommodity flow}. Masalah ini dikenal memiliki kompleksitas \textit{NP-complete}. 

Dalam praktiknya, mengalokasikan \textit{bandwidth} secara penuh untuk setiap koneksi tidak dapat dilakukan tidak selalu memungkinkan, terlepas dari bagaimana VM ditempatkan dalam PM. Oleh karena itu, salah satu objektif dalam model optimasi ini adalah memaksimalkan total bandwidth yang dapat dialokasikan dalam jaringan. Dalam konteks masalah multicommodity flow, objektif ini berkaitan dengan masalah \textit{maximum flow}. Selain itu, meminimalkan \textit{delay} komunikasi antar PM juga menjadi objektif penting model ini untuk meningkatkan efisiensi transmisi data. Dalam konteks masalah multicommodity flow, objektif ini berkaitan dengan masalah \textit{minimum cost}.

Dalam infrastruktur \textit{cloud data center} modern, teknologi \textit{Software Defined Networking} (SDN) memungkinkan pengelolaan jaringan yang lebih fleksibel dan terpusat. SDN memisahkan \textit{control plane} dari \textit{data plane}, sehingga kontrol lalu lintas jaringan dapat dilakukan melalui \textit{controller} yang memiliki visibilitas penuh terhadap topologi jaringan. Dengan pendekatan ini, rute komunikasi antar \textit{Physical Machine} (PM) dapat ditentukan secara eksplisit dan dikonfigurasi sesuai kebutuhan. 

Selain itu, teknologi \textit{Multipath TCP} (MP-TCP) memungkinkan pemisahan lalu lintas komunikasi ke dalam beberapa rute secara bersamaan (Ford dkk., 2011). Beberapa jaringan yang berhasil mengimplementasikan \textit{multicommodity flow} sebagai solusi \textit{traffic engineering} dalam menggunakan SDN adalah B4, jaringan WAN pribadi milik Google (Hong dkk., 2018), dan SWAN, jaringan WAN pribadi milik Microsoft (Hong dkk., 2013).

Oleh karena itu, asumsi dalam model yang diajukan—yakni bahwa rute komunikasi dapat ditentukan dan lalu lintas komunikasi dapat dibagi menjadi  lebih dari satu rute —sejalan dengan praktik yang diterapkan dalam \textit{cloud data center} berbasis SDN.
