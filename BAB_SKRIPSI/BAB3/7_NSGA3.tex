\textit{Nondominated Sorted Genetic Algorithm} III (NSGA-III) merupakan versi ketiga NSGA yang dikembangkan oleh Deb dan Jain (2014). Ketiga versi algoritma ini memiliki kerangka kerja yang serupa, terutama dalam penggunaan teknik \textit{non-dominated sorting} (pengurutan takterdominasi) untuk mengevaluasi populasi yang dihasilkan. \textit{Non-dominated sorting} adalah metode pemeringkatan individu sebagai solusi atau vektor objektif dalam optimasi multiobjektif berdasarkan prinsip dominasi Pareto. Seperti dijelaskan pada subbab sebelumnya, individu $\mathbf{x}^{(1)}$ dikatakan mendominasi individu $\mathbf{x}^{(2)}$ jika dan hanya jika tidak ada satu pun objektif pada $\mathbf{x}^{(1)}$ yang lebih buruk daripada objektif yang sama pada $\mathbf{x}^{(2)}$, serta terdapat setidaknya satu objektif pada $\mathbf{x}^{(1)}$ yang lebih baik dibandingkan objektif yang sama pada $\mathbf{x}^{(2)}$. Dalam hal ini, individu $\mathbf{x}^{(1)}$ memiliki peringkat lebih tinggi daripada individu $\mathbf{x}^{(2)}$. Individu-individu dengan peringkat yang sama dikelompokkan ke dalam \textit{front} takterdominasi yang sama. Selain itu, individu dalam \textit{front} takterdominasi yang sama memiliki setidaknya dua objektif yang saling berkonflik, sehingga tidak ada individu yang mendominasi solusi lainnya.

Misalkan \textit{front} takterdominasi peringkat $i$ dalam populasi $P$ dilambangkan sebagai $F_i \subseteq P$. \textit{Front} pertama, $F_1$ berisi semua solusi yang tidak didominasi oleh solusi mana pun di $P$. Selanjutnya, \textit{front} $F_2$ berisi semua solusi takterdominasi pada $P \setminus F_1$. Demikian pula dengan \textit{front} $F_3$ yang terdiri dari semua solusi takterdominasi pada $P \setminus (F_1 \cup F_2)$, dan seterusnya. Secara umum, untuk $i > 1$, $F_i$ adalah himpunan solusi takterdominasi dalam $P\setminus(\bigcup_{j=1}^{i-1} F_j)$.  

Sebagai contoh, misalkan $P = \{\mathbf{x}^{(1)},\mathbf{x}^{(2)},\mathbf{x}^{(3)},\mathbf{x}^{(4)},\mathbf{x}^{(5)},\mathbf{x}^{(6)}\}$ adalah populasi dengan enam individu untuk suatu masalah optimasi dengan tiga objektif, di mana 

$$
\begin{align}
  \mathbf{z}^{(1)}=\mathbf{f}(\mathbf{x}^{(1)})=\begin{bmatrix}0.67228 & 0.29762 & 0.37744\end{bmatrix}\\
  \mathbf{z}^{(2)}=\mathbf{f}(\mathbf{x}^{(2)})=\begin{bmatrix}0.48808 & 0.04670 & 0.49415\end{bmatrix}\\
  \mathbf{z}^{(3)}=\mathbf{f}(\mathbf{x}^{(3)})=\begin{bmatrix}0.82550 & 0.99063 & 0.92895\end{bmatrix}\\
  \mathbf{z}^{(4)}=\mathbf{f}(\mathbf{x}^{(4)})=\begin{bmatrix}0.03145 & 0.00683 & 0.39545\end{bmatrix}\\
  \mathbf{z}^{(5)}=\mathbf{f}(\mathbf{x}^{(5)})=\begin{bmatrix}0.80805 & 0.76979 & 0.97396\end{bmatrix}\\
  \mathbf{z}^{(6)}=\mathbf{f}(\mathbf{x}^{(6)})=\begin{bmatrix}0.56562 & 0.74677 & 0.52441\end{bmatrix}\\
\end{align}
$$

Berikut adalah relasi dominasi antar individu di $P$:

\begin{itemize}
  \item $ \mathbf{z}^{(3)} \succ \mathbf{z}^{(1)} $
  \item $ \mathbf{z}^{(5)} \succ \mathbf{z}^{(1)} $
  \item $ \mathbf{z}^{(3)} \succ \mathbf{z}^{(2)} $
  \item $ \mathbf{z}^{(5)} \succ \mathbf{z}^{(2)} $
  \item $ \mathbf{z}^{(6)} \succ \mathbf{z}^{(2)} $
  \item $ \mathbf{z}^{(2)} \succ \mathbf{z}^{(4)} $
  \item $ \mathbf{z}^{(3)} \succ \mathbf{z}^{(4)} $
  \item $ \mathbf{z}^{(5)} \succ \mathbf{z}^{(4)} $
  \item $ \mathbf{z}^{(6)} \succ \mathbf{z}^{(4)} $
  \item $ \mathbf{z}^{(3)} \succ \mathbf{z}^{(6)} $
  \item $ \mathbf{z}^{(5)} \succ \mathbf{z}^{(6)} $
\end{itemize}

Perhatikan bahwa $\mathbf{x}^{(1)}$ dan $\mathbf{x}^{(4)}$ tidak didominasi oleh individu manapun di $P$. Oleh karena itu, $\mathbf{x}^{(1)}$ dan $\mathbf{x}^{(4)}$ termasuk ke dalam front takterdominasi peringkat pertama. Selain itu, $\mathbf{x}^{(3)}$ dan $\mathbf{x}^{(5)}$ tidak mendominasi individu manapun, sehingga $\mathbf{x}^{(3)}$ dan $\mathbf{x}^{(5)}$ termasuk ke dalam front takterdominasi peringkat terakhir. Dengan demikian,

\begin{itemize}  
  \item $F_1 = \{\mathbf{x}^{(1)}, \mathbf{x}^{(4)}\}$
  \item $F_2 = \{\mathbf{x}^{(2)}\}$
  \item $F_3 = \{\mathbf{x}^{(6)}\}$
  \item $F_4 = \{\mathbf{x}^{(3)}, \mathbf{x}^{(5)}\}$
\end{itemize}

Berbeda dengan NSGA-II yang menggunakan \textit{crowding distance} untuk mempertahankan keragaman individu dalam populasi \citep{Deb2002}, NSGA-III mengandalkan sekumpulan titik referensi (\textit{reference points}). Titik-titik referensi ini dapat disesuaikan dengan karakteristik masalah yang diselesaikan. Jika tidak ada informasi mengenai preferensi solusi dari pengambil keputusan, titik-titik ini dapat ditentukan secara sistematis menggunakan metode yang disarankan oleh \citet{DasDennis1998}. Titik-titik referensi dikonstruksi sekali di awal eksekusi NSGA-III dan digunakan secara konsisten di setiap generasi sebagai tolok ukur keragaman populasi.
