Menurut National Institute of Standards and Technology, "\textit{cloud computing is a model for enabling ubiquitous, convenient, on-demand network access to a shared pool of configurable computing resources (e.g., networks, servers, storage, applications, and services) that can be rapidly provisioned and released with minimal management effort or service provider interaction}." \citet{MellGrance2011}. Dalam bahasa Indonesia, \textit{cloud computing} (komputasi awan) adalah model komputasi di mana sumber daya komputasi (misalnya, jaringan, server, penyimpanan, aplikasi, dan layanan) dapat diakses melalui jaringan dari mana saja secara praktis sesuai permintaan. Sumber daya tersebut dapat dikonfigurasi, disediakan, dan dilepaskan dalam waktu singkat dengan upaya pengelolaan serta interaksi dengan penyedia layanan yang minimal. 

\textit{Infrastructure as a Service} (IaaS) adalah salah satu dari tiga model layanan \textit{cloud computing}, selain \textit{Platform as a Service} (PaaS) dan \textit{Software as a Service} (SaaS) \citet{MellGrance2011}. IaaS menyediakan akses daya komputasi, penyimpanan, jaringan, serta sumber daya komputasi lainnya dalam bentuk abstrak atau virtual. Melalui abstraksi, pengguna dapat menyewa sumber daya tanpa perlu memiliki, mengelola, merawatnya sumber daya fisiknya secara langsung. Teknologi virtualisasi menjadi kunci terwujudnya abstraksi ini. Virtualisasi adalah proses yang mengemulasikan sumber daya komputasi fisik ke dalam bentuk abstrak atau virtual. Melalui virtualisasi, penyedia layanan dapat menawarkan akses ke berbagai sumber daya komputasi, seperti mesin virtual (\textit{virtual machine} atau VM) yang memungkinkan pengguna menjalankan sistem operasi dan aplikasi seperti pada komputer fisik.
 
Kemampuan memvirtualisasi perangkat keras fisik ini dimungkinkan oleh \textit{hypervisor}, sebuah perangkat lunak yang menjadi perantara bagi mesin virtual dalam mengakses dan berkomunikasi dengan perangkat keras milik mesin fisik (\textit{physical machine} atau PM) tempat mesin virtual tersebut berjalan. Mesin virtual diinstal di atas \textit{hypervisor} yang diinstal di dalam mesin fisik. \textit{Hypervisor} berperan memetakan sumber daya fisik pada PM kepada VM yang berjalan di dalamnya sesuai dengan kebutuhan. Selain itu, \textit{hypervisor} juga memungkinkan PM tersebut menjalankan lebih dari satu VM dengan secara konkuren, meskipun menggunakan sistem operasi yang berbeda-beda. Meskipun demikian, \textit{hypervisor} memastikan setiap VM terisolasi satu sama lain sehingga masing-masing dapat menjalankan program dan aplikasinya masing-masing tanpa terdampak oleh masalah dan ketidakstabilan yang terjadi pada VM lain di PM yang sama \citep{Hill2013}. Hal ini dapat meningkatkan \textit{fault tolerance}, kemampuan sistem untuk tetap beroperasi meskipun galat atau malfungsi. 

Mesin virtual dibentuk dari sebuah \textit{image} atau \textit{template} yang menentukan spesifikasi seperti \textit{virtual} CPU (vCPU), \textit{virtual} RAM, \textit{virtual disk}, serta \textit{image file} dari disk tersebut. \textit{Image} mesin virtual berupa \textit{file} yang disimpan pada \textit{disk}, sehingga dapat dengan mudah dibuat, dihapus, disalin, atau ditransfer ke mesin lain \citep{Huawei}. Pengguna juga dapat mengambil \textit{snapshot} VM kapan saja untuk menyimpan konfigurasi dan lingkungan (\textit{environment}) VM yang sedang berjalan. \textit{Snapshot} ini memungkinkan pengguna memulihkan atau menjalankan ulang lingkungan yang telah disimpan sebelumnya.

Migrasi dan replikasi VM dapat dilakukan dengan mentransfer atau menyalin \textit{image} VM tersebut \citep{Huawei}. Kemampuan migrasi VM mendukung peningkatan ketersediaan (\textit{availability}), penghindaran bencana (\textit{disaster avoidance}), serta pemulihan dari bencana (\textit{disaster recovery}). Sementara itu, kemampuan replikasi VM mendukung peningkatan ketersediaan dan skalabilitas \citep{Hill2013}. Dengan demikian, virtualisasi meningkatkan ketersediaan, penghindaran dan pemulihan dari bencana, serta skalabilitas aplikasi yang di-\textit{deploy} oleh pengguna.

Kemampuan virtualisasi ini mendukung pendekatan \textit{multi-tenancy}, yaitu penggunaan infrastruktur komputasi yang sama oleh beberapa pengguna sekaligus \citep{Cloudflare}. \textit{Multi-tenancy} dapat ditunjukkan oleh kemampuan lebih dari satu VM untuk berjalan dan mengakses sumber daya pada PM yang sama. Pendekatan \textit{multi-tenancy} dapat mengurangi kebutuhan server pada \textit{cloud data center} karena penyedia layanan cloud tidak perlu menyediakan satu server fisik untuk setiap pengguna \citep{Hill2013}. Berkurangnya kebutuhan server juga mengurangi konsumsi energi listrik oleh \textit{data center} penyedia layanan. Selain itu, \textit{multi-tenancy} memungkinkan pemanfaatan sumber daya komputasi yang lebih efisien dibandingkan pendekatan \textit{single-tenancy} \citep{Cloudflare}. Dalam pendekatan \textit{single-tenancy}, setiap pengguna disediakan akses ke server tersendiri (\textit{dedicated server}). Pendekatan ini tidak hanya membutuhkan lebih banyak server pada \textit{cloud data center}, tetapi juga menyebabkan sumber daya tidak digunakan secara optimal. Sumber daya yang dialokasikan untuk satu pengguna tidak dapat diakses oleh pengguna lain, sehingga banyak sumber daya yang tidak terpakai. 
