% ------------------------------------------------------------------------------
% Berisi tambahan package dan konfigurasi untuk masing-masing package.
% Ada baiknya, setiap konfigurasi diletakkan tepat dibawah 
% setelah package dilakukan import (usepackage) agar tidak membingungkan.
% Serta disarankan untuk menambah kegunaan package tersebut agar tidak lupa.
% ------------------------------------------------------------------------------

% font tambahan
\usepackage{textcomp}

% digunakan untuk membuat flowchart
\usepackage{tikz}
\usetikzlibrary{shapes, shapes.misc, arrows, fit, positioning}
\tikzstyle{block} = [rectangle, draw, fill=gray!20, text width=4cm, text centered, rounded corners, minimum height=3em]
\tikzstyle{io} = [trapezium, draw, fill=gray!20, text width=1cm, text centered, rounded corners, minimum height=3em]
\tikzstyle{decision} = [diamond, draw, fill=gray!20, text width=3cm, text centered, minimum height=3em]

\usepackage{float}
\usepackage{booktabs}
\usepackage{pbox}
\usepackage{multirow}
\usepackage[normalem]{ulem}
\useunder{\uline}{\ul}{}

% Untuk hyperlink dan otomatis membuat bookmark
\usepackage{hyperref}

% break tanda /, - dan spasi ke baris baru jika sudah tidak muat
\def\UrlBreaks{\do\/\do-\do\ }

% font url dibuat miring dan dg jenis font ttfamily
\renewcommand{\UrlFont}{\small\ttfamily\itshape}

\usepackage{csquotes}
\usepackage{framed}
\usepackage{enumitem}

% untuk input kode baik dari file atau bukan
\usepackage{listings}

% ----------------------------------------------------------------------------
% Contoh dari file
% ----------------------------------------------------------------------------
% \begin{figure}[H]
%   \lstinputlisting[language=python, firstline=38, lastline=59]{code/linkwalker.py}
%   \caption{Mendapatkan daftar tautan berita pada kompas.com}
%   \label{grab daftar berita kompas}
% \end{figure}
% ----------------------------------------------------------------------------
%
% ----------------------------------------------------------------------------
% Contoh
% ----------------------------------------------------------------------------
% \begin{figure}
% 	\begin{lstlisting}[language=sql]
% 		update train_data_statement set data = replace(data, '“', '"');
% 		update train_data_statement set data = replace(data, '”', '"');

% 		update test_data_statement set data = replace(data, '“', '"');
% 		update test_data_statement set data = replace(data, '”', '"');
% 	\end{lstlisting}
% 	\caption{\textit{Query} SQL untuk melakukan perubahan karakter pada data}
% 	\label{kueri SQL untuk melakukan perubahan karakter pada data}
% \end{figure}
% ------------------------------------------------------------------------------

\usepackage{color}
\usepackage{amsmath}
\usepackage{courier}
\usepackage[scaled=.75]{beramono}

%-----------------------------------------------------------------
% Setting syntax hightlighting
%-----------------------------------------------------------------
\lstset{frame=tb,
  language=Python,
  aboveskip=2mm,
  belowskip=1mm,
  showstringspaces=false,
  columns=flexible,
  basicstyle  = \fontfamily{pcr}\fontsize{8pt}{8pt}\selectfont,
  numbersep=8pt,
  numbers=left,
  numberstyle=\tiny\color{gray},
  keywordstyle=\color{blue},
  commentstyle=\color{dkgreen},
  stringstyle=\color{mauve},
  breaklines=true,
  breakatwhitespace=true,
  tabsize=4
}

% Untuk menghilangkan titik-titik pada daftar isi

\usepackage[titles]{tocloft}
\renewcommand{\cftdot}{}

% Untuk membuat multi kolom
\usepackage{etoolbox,refcount}
\usepackage{multicol}

% Konfigurasi multi kolom
% bikin multi kolom
\newcounter{countitems}
\newcounter{nextitemizecount}
\newcommand{\setupcountitems}{%
  \stepcounter{nextitemizecount}%
  \setcounter{countitems}{0}%
  \preto\item{\stepcounter{countitems}}%
}
\makeatletter
\newcommand{\computecountitems}{%
  \edef\@currentlabel{\number\c@countitems}%
  \label{countitems@\number\numexpr\value{nextitemizecount}-1\relax}%
}
\newcommand{\nextitemizecount}{%
  \getrefnumber{countitems@\number\c@nextitemizecount}%
}
\newcommand{\previtemizecount}{%
  \getrefnumber{countitems@\number\numexpr\value{nextitemizecount}-1\relax}%
}
\makeatother
\newenvironment{AutoMultiColItemize}{%
\ifnumcomp{\nextitemizecount}{>}{2}{\begin{multicols}{2}}{}%
\setupcountitems\begin{itemize}}%
{\end{itemize}%
\unskip\computecountitems\ifnumcomp{\previtemizecount}{>}{2}{\end{multicols}}{}}
%end bikin multi kolom

% ------------------------------------------------------------------------------
% Contoh sintaks:
% ------------------------------------------------------------------------------
% \begin{itemize}
%   \item \textit{Reporting verb} yang hadir sebelum entitas pada kutipan langsung:
%   \begin{AutoMultiColItemize}
% 	  \item tutur
% 	  \item kata
% 	  \item ujar
%   \end{AutoMultiColItemize}

%   \item \textit{Reporting verb} yang hadir setelah entitas pada kutipan langsung:
%   \begin{AutoMultiColItemize}
% 	  \item mengatakan
% 	  \item menjawab
%   \end{AutoMultiColItemize}
% \end{itemize}
% ------------------------------------------------------------------------------


% Setting list agar spasi antar list tidak terlalu banyak
\setlist{
listparindent=\parindent,
parsep=0pt
}

% Agar tetap Justify tapi kata tidak dipisah sesuka hati (not hypenation but justified)
\tolerance=1
\emergencystretch=\maxdimen
\hyphenpenalty=10000
\hbadness=10000
\hyphenchar\font=-1
\sloppy

% Agar support longtable
% https://tex.stackexchange.com/questions/639452/create-long-table-in-latex
% need: varwidth, ninecolors
%%%% begin of required preamble
\usepackage{tabularray}
\UseTblrLibrary{varwidth}
\DefTblrTemplate{contfoot-text}{default}{Bersambung di halaman selanjutnya}
\DefTblrTemplate{conthead-text}{default}{(Lanjutan)}

% https://stackoverflow.com/a/16804893
% \usepackage{tablefootnote}
% \usepackage{footnote}
% \makesavenoteenv{tabular}
% \makesavenoteenv{table}

\usepackage{amsmath}   % <-- for \eqref

\usepackage{pgfgantt}

% for \includepdf
\usepackage{pdfpages}
