
\subsection{Key Filtering}
\begingroup

{\small \emph{An extension by Christian Feuersänger}}
\vspace{0.4cm}%

\noindent Normally, a call to |\pgfkeys| sets all keys provided in its argument
list. This is usually what users expect it to do. However, implementations of
different packages or \pgfname-libraries may need more control over the key
setting procedure: library~A may want to set its options directly and
communicate all remaining ones to library~B.

\begin{pgfkeyslibrary}{filtered}
  The |filtered| library provides functionality to filter keys and only process
  selected ones, while forwarding the others to another namespace.

  This library is loaded by default by |pgfkeys|.
\end{pgfkeyslibrary}

This section describes key filtering methods of \pgfname, including options for
family groupings. If you merely want to use \pgfname\ (or its libraries), you
can skip this section. It is addressed to package (or library) authors.

\subsubsection{Starting With An Example}
\label{section-key-filter-example}

Users of |xkeyval| are familiar with the concept of key families: keys belong
to groups and those keys can be `filtered' out of other options. \pgfname\
supports family groupings and more abstract key selection mechanism with
|\pgfkeysfiltered|, a variant of |\pgfkeys|. Suppose we have the example key
grouping

\begin{codeexample}[setup code,hidden]
\pgfkeys{
    /my group/A1/.code=(A1:#1),
    /my group/A2/.code=(A2:#1),
    /my group/A3/.code=(A3:#1),
    /my group/B/.code=(B:#1),
    /my group/C/.code=(B:#1),
    /my group/A/.is family,
    /my group/A1/.belongs to family=/my group/A,
    /my group/A2/.belongs to family=/my group/A,
    /my group/A3/.belongs to family=/my group/A,
    /pgf/key filters/active families/.install key filter,
    /my group/A/.activate family,
}
\end{codeexample}
%
\begin{codeexample}[code only]
\pgfkeys{
    /my group/A1/.code=(A1:#1),
    /my group/A2/.code=(A2:#1),
    /my group/A3/.code=(A3:#1),
    /my group/B/.code=(B:#1),
    /my group/C/.code=(B:#1),
}
\end{codeexample}

\noindent and we want to set options |A1|, |A2| and |A3| only. A call to
|\pgfkeys| yields
%
\begin{codeexample}[]
\pgfkeys{/my group/A1=a1, /my group/A2=a2, /my group/B=b,  /my group/C=c}
\end{codeexample}
%
\noindent because all those command option are processed consecutively.

Now, let's define a family named |A| which contains |A1|, |A2| and |A3| and set
only family members of |A|. We prepare our key settings with
%
\begin{codeexample}[code only]
\pgfkeys{
    /my group/A/.is family,
    /my group/A1/.belongs to family=/my group/A,
    /my group/A2/.belongs to family=/my group/A,
    /my group/A3/.belongs to family=/my group/A,
}
\end{codeexample}
%
\noindent and
%
\begin{codeexample}[code only,preamble={\usepgfkeyslibrary{filtered}}]
\pgfkeys{/pgf/key filters/active families/.install key filter}
\end{codeexample}
%
\noindent After this preparation, we can use |\pgfkeysfiltered| with
%
\begin{codeexample}[preamble={\usepgfkeyslibrary{filtered}}]
\pgfkeys{/my group/A/.activate family}
\pgfkeysfiltered{/my group/A1=a1, /my group/A2=a2,
 /my group/B=b,  /my group/C=c}
\end{codeexample}
%
\noindent or
%
\begin{codeexample}[preamble={\usepgfkeyslibrary{filtered}}]
\pgfkeys{/my group/A/.activate family}
\pgfkeysfiltered{/my group/A1=a1, /my group/A2=a2,
 /my group/B=b,  /my group/C=c, /tikz/color=blue, /my group/A3=a3}
\end{codeexample}
%
\noindent to set only keys which belong to an `active' family -- in our case,
only family~|A| was active, so the remaining options have not been processed.
The family processing is quite fast and allows an arbitrary number of active
key families.

Unprocessed options can be collected into a macro (similar to |xkeyval|'s
|\xkv@rm|), discarded or handled manually. The details of key selection and
family declaration are described in the following sections.


\subsubsection{Setting Filters}

The command |\pgfkeysfiltered| is the main tool to process only selected
options. It works as follows.
%
\begin{command}{\pgfkeysfiltered\marg{key--value-list}}
    Processes all options in exactly the same way as
    |\pgfkeys|\marg{key--value-list}, but a key filter is considered as soon as
    key identification is complete.

    The key filter tells |\pgfkeysfiltered| whether it should continue to apply
    the current option (return value is `true') or whether something different
    shall be done (filter returns `false').

    There is exactly one key filter in effect, and it is installed by the
    |.install key filter| handler or by |\pgfkeysinstallkeyfilter|.

    If the key filter returns `false', a unique key filter handler gets
    control. This handler is installed by the |.install key filter handler|
    method and has access to the key's full name, value and (possibly) path.

    Key filtering applies to any (possibly nested) call to |\pgfkeys|,
    |\pgfkeysalso|, |\pgfqkeys| and |\pgfqkeysalso| during the evaluation of
    \marg{key--value-list}. It does \emph{not} apply to routines like
    |\pgfkeyssetvalue| or |\pgfkeysgetvalue|. Furthermore, keys belonging to
    |/errors| are always processed. Key filtering routines can't be nested: you
    can't combine different key filters automatically.
\end{command}

\begin{command}{\pgfqkeysfiltered\marg{default-path}\marg{key--value-list}}
    A variant of |\pgfkeysfiltered| which uses the `quick' search path setting.
    It is the |\pgfqkeys| variant of |\pgfkeysfiltered|, see the documentation
    for |\pgfqkeys| for more details.
\end{command}

\begin{command}{\pgfkeysalsofrom\marg{macro}}
    A variant of |\pgfkeysalso| which loads its key list from \marg{macro}.

    It is useful in conjunction with the
    |/pgf/key filter handlers/append filtered to=|\meta{macro} handler.

    The following example uses the same settings as in the intro
    section~\ref{section-key-filter-example}.
    %
\begin{codeexample}[preamble={\usepgfkeyslibrary{filtered}}]
\pgfkeys{/pgf/key filter handlers/append filtered to/.install key filter handler=\remainingoptions}
\def\remainingoptions{}
\pgfkeysfiltered{/my group/A1=a1, /my group/A2=a2,
 /my group/B=b,  /my group/C=c, /tikz/color=blue, /my group/A3=a3}

Remaining: `\remainingoptions'.
\pgfkeysalsofrom{\remainingoptions}
\end{codeexample}
    %
\end{command}

\begin{command}{\pgfkeysalsofiltered\marg{key--value-list}}
    This command works as |\pgfkeysfiltered|, but it does not change the
    current default path. See the documentation of |\pgfkeysalso| for more
    details.
\end{command}

\begin{command}{\pgfkeysalsofilteredfrom\marg{macro}}
    A variant of |\pgfkeysalsofiltered| which loads its key list from
    \marg{macro}.
\end{command}

\begin{handler}{{.install key filter}|=|\meta{optional arguments}}
    This handler installs a key filter. A key filter is a command key which
    sets the \TeX-boolean |\ifpgfkeysfiltercontinue|, that means a key with
    existing `|/.@cmd|' suffix. A simple example is a key filter which returns
    always true:
    %
\begin{codeexample}[code only,preamble={\usepgfkeyslibrary{filtered}}]
\pgfkeys{/foo/bar/true key filter/.code={\pgfkeysfiltercontinuetrue}}
\pgfkeys{/foo/bar/true key filter/.install key filter}
\end{codeexample}
    %
    If key filters require arguments, they are installed by
    |.install key filter| as well. An example is the |/pgf/key filters/equals|
    handler:
    %
\begin{codeexample}[preamble={\usepgfkeyslibrary{filtered}}]
\pgfkeys{/pgf/key filters/equals/.install key filter={/my group/A1}}
\pgfkeysfiltered{/my group/A1=a1, /my group/A2=a2,
/my group/B=b,  /my group/C=c, /tikz/color=blue, /my group/A3=a3}
\end{codeexample}
    %
    If a key filter requires more than one argument, you need to provide the
    complete argument list in braces like |{{first}{second}}|.

    You can also use |\pgfkeysinstallkeyfilter|\meta{full key}\meta{optional
    arguments}, it has the same effect.

    See section~\ref{section-key-writing-filters} for how to write key filters.
\end{handler}

\begin{handler}{{.install key filter handler}|=|\meta{optional arguments}}
    This handler installs the routine which will be invoked for every
    \emph{unprocessed} option, that means any option for which the key filter
    returned `false'.

    The |.install key filter handler| is used in the same way as
    |.install key filter|. There exists a macro version,
    |\pgfkeysinstallkeyfilterhandler|\meta{full key}\meta{optional arguments},
    which has the same effect.

    See section~\ref{section-key-writing-filters} for how to write key filter
    handlers.
\end{handler}


\subsubsection{Handlers For Unprocessed Keys}

Each option for which key filters decided to skip them is handed over to a `key
filter handler'. There are several predefined key filter handlers.

\begin{key}{/pgf/key filter handlers/append filtered to=\marg{macro}}
    Install this filter handler to append any unprocessed options to macro
    \marg{macro}.
    %
\begin{codeexample}[preamble={\usepgfkeyslibrary{filtered}}]
\pgfkeys{/pgf/key filter handlers/append filtered to/.install key filter handler=\remainingoptions}
\def\remainingoptions{}
\pgfkeysfiltered{/my group/A1=a1, /my group/A2=a2,
 /my group/B=b,  /my group/C=c, /tikz/color=blue}

Remaining options: `\remainingoptions'.
\end{codeexample}
    %
    This example uses the same keys as defined in the intro
    section~\ref{section-key-filter-example}.
\end{key}

\begin{key}{/pgf/key filter handlers/ignore}
    Install this filter handler if you simply want to ignore any unprocessed
    option. This is the default.
\end{key}

\begin{key}{/pgf/key filter handlers/log}
    This key filter handler writes messages for any unprocessed option to your
    logfile (and terminal).
\end{key}


\subsubsection{Family Support}

\pgfname{} supports a family concept: every option can be associated with (at
most) one family. Families form loose key groups which are independent of the
key hierarchy. For example, |/my tree/key1| can belong to family |/tikz|.

It is possible to `activate' or `deactivate' single families. Furthermore, it
is possible to set only keys which belong to active families using appropriate
key filter handlers.

The family support is fast: if there are $N$ options in a key--value-list and
there are $K$ active families, the runtime for |\pgfkeysfiltered| is $O(N+K)$
(activate every family $O(K)$, check every option $O(N)$, deactivate every
family $O(K)$).

\begin{handler}{{.is family}}
    Defines a new family. This option has already been described in
    section~\ref{section-is family-handler} on page~\pageref{section-is
    family-handler}.
\end{handler}

\begin{handler}{{.activate family}}
    Activates a family. The family needs to be defined, otherwise
    |/errors/family unknown| will be raised.

    Activation means a \TeX-boolean will be set to |true|, indicating that a
    family should be processed.

    You can also use |\pgfkeysactivatefamily|\meta{full path} to get the same
    effect. Furthermore, you can use |\pgfkeysactivatefamilies|\meta{list of
    families}\meta{macro name for de-activation} to activate a list of families
    (see section~\ref{section-key-filter-api}).
\end{handler}

\begin{handler}{{.deactivate family}}
    Deactivates a family. The family needs to be defined, otherwise
    |/errors/family unknown| will be raised.

    You can also use |\pgfkeysdeactivatefamily|\meta{full path} to get the same
    effect.
\end{handler}

\begin{handler}{{.belongs to family}|=|\marg{family name}}
    Associates the current option with \marg{family name}, which is expected to
    be a full path of a family.
    %
\begin{codeexample}[code only,preamble={\usepgfkeyslibrary{filtered}}]
\pgfkeys{/foo/bar/.is family}
\pgfkeys{
    /foo/a/.belongs to family=/foo/bar,
    /foo/b/.belongs to family=/foo/bar
}
\end{codeexample}
    %
    Each option can have up to one family, |.belongs to family| overwrites any
    old setting.
\end{handler}

\begin{key}{/pgf/key filters/active families}
    Install this key filter if |\pgfkeysfiltered| should only process activated
    families. If a key does not belong to any family, it is not processed. If a
    key is completely unknown within the default path, the normal `unknown'
    handlers of |\pgfkeys| are invoked.
\end{key}

\begin{key}{/pgf/key filters/active families or no family=\marg{key filter 1}\marg{key filter 2}}
    This key filter configures |\pgfkeysfiltered| to work as follows.
    %
    \begin{enumerate}
        \item If the current key belongs to a family, set
            |\ifpgfkeysfiltercontinue| to true if and only if its family is
            active.
        \item If the current key does \emph{not} belong to a family, assign
            |\ifpgfkeysfiltercontinue| as result of \marg{key filter 1}.
        \item If the current key is unknown within the default path, assign
            |\ifpgfkeysfiltercontinue| as result of \marg{key filter 2}.
    \end{enumerate}
    %
    The arguments \marg{key filter 1} and \marg{key filter 2} are other key
    filters (possibly with options) and allow fine-grained control over the
    filtering process.
    %
\begin{codeexample}[code only,preamble={\usepgfkeyslibrary{filtered}}]
\pgfkeysinstallkeyfilter
    {/pgf/key filters/active families or no family}
    {{/pgf/key filters/is descendant of=/tikz}% for keys without family
     {/pgf/key filters/false}% for unknown keys
    }%
\end{codeexample}
    %
    This key filter will return true for any option with active family. If an
    option has no family, the return value is true if and only if it belongs to
    |/tikz|. If the option is unknown, the return value is |false| and unknown
    handlers won't be called.
\end{key}

\begin{key}{/pgf/key filters/active families or no family DEBUG=\marg{key filter 1}\marg{key filter 2}}
    A variant of |active families or no family| which protocols each action on
    your terminal (log-file).
\end{key}

\begin{key}{/pgf/key filters/active families and known}
    A fast alias for

    |/pgf/key filters/active families or no family=|\par
    \noindent\hskip10pt|{/pgf/keys filters/false}|\par
    \noindent\hskip10pt|{/pgf/keys filters/false}|.
\end{key}
%
\begin{key}{/pgf/key filters/active families or descendants of=\marg{path prefix}}
    A fast alias for

    |/pgf/key filters/active families or no family=|\par
    \noindent\hskip10pt|{/pgf/keys filters/is descendant of=|\marg{path prefix}|}|\par
    \noindent\hskip10pt|{/pgf/keys filters/false}|.
\end{key}

\begin{command}{\pgfkeysactivatefamiliesandfilteroptions\marg{family list}\marg{key--value-list}}
    A simple shortcut macro which activates any family in the comma separated
    \marg{family list}, invokes |\pgfkeysfiltered|\meta{key--value-list} and
    deactivates the families afterwards.

    Please note that you will need to install a family key filter, otherwise
    family activation has no effect.
\end{command}
%
\begin{command}{\pgfqkeysactivatefamiliesandfilteroptions\marg{family list}\marg{default path}\marg{key--value-list}}
    The `quick' default path variant of |\pgfkeysactivatefamiliesandfilteroptions|.
\end{command}

\begin{command}{\pgfkeysactivatesinglefamilyandfilteroptions\marg{family name}\marg{key--value-list}}
    A shortcut macro which activates a single family and invokes |\pgfkeysfiltered|.

    Please note that you will need to install a family key filter, otherwise
    family activation has no effect.
\end{command}
%
\begin{command}{\pgfqkeysactivatesinglefamilyandfilteroptions\marg{family name}\marg{default path}\marg{key--value-list}}%
    The `quick' default path variant of |\pgfkeysactivatesinglefamilyandfilteroptions|.
\end{command}


\subsubsection{Other Key Filters}

There are some more key filters which have nothing to do with family handling.
%
\begin{key}{/pgf/key filters/is descendant of=\marg{path}}
    Install this key filter to process only options belonging to the key tree
    \meta{path}. It returns true for every key whose key path is equal to
    \meta{path}. It also returns true for any unknown key, that means unknown
    keys are processed using the standard unknown handlers of \pgfname.
    %
\begin{codeexample}[preamble={\usepgfkeyslibrary{filtered}}]
\pgfkeys{
/group 1/A/.code={(A:#1)},
/group 1/foo/bar/B/.code={(B:#1)},
/group 2/C/.code={(C:#1)},
/pgf/key filters/is descendant of/.install key filter=/group 1}
\pgfkeysfiltered{/group 1/A=a,/group 1/foo/bar/B=b,/group 2/C=c}
\end{codeexample}
    %
\end{key}

\begin{key}{/pgf/key filters/equals=\marg{full key}}
    Install this key filter to process only the fully qualified option
    \marg{full key}. The filter returns true for any unknown key or if the key
    equals \marg{full key}.
    %
\begin{codeexample}[preamble={\usepgfkeyslibrary{filtered}}]
\pgfkeys{
/group 1/A/.code={(A:#1)},
/group 1/B/.code={(B:#1)},
/pgf/key filters/equals/.install key filter=/group 1/A}
\pgfqkeysfiltered{/group 1}{A=a,B=b}
\end{codeexample}
    %
\end{key}

\begin{key}{/pgf/key filters/not=\marg{key filter}}
    This key filter logically inverts the result of \marg{key filter}.
    %
\begin{codeexample}[preamble={\usepgfkeyslibrary{filtered}}]
\pgfkeys{
/group 1/A/.code={(A:#1)},
/group 1/foo/bar/B/.code={(B:#1)},
/group 2/C/.code={(C:#1)},
/pgf/key filters/not/.install key filter=
    {/pgf/key filters/is descendant of=/group 1}}
\pgfkeysfiltered{/group 1/A=a,/group 1/foo/bar/B=b,/group 2/C=c}
\end{codeexample}
    %
    Please note that unknown keys will be handed to the usual unknown handlers.
\end{key}

\begin{key}{/pgf/key filters/and=\marg{key filter 1}\marg{key filter 2}}
    This key filter returns true if and only if both, \marg{key filter 1} and
    \marg{key filter 2} return true.
\end{key}

\begin{key}{/pgf/key filters/or=\marg{key filter 1}\marg{key filter 2}}
    This key filter returns true if one of \marg{key filter 1} and \marg{key
    filter 2} returns true.
\end{key}

\begin{key}{/pgf/key filters/true}
    This key filter returns always true.
\end{key}

\begin{key}{/pgf/key filters/false}
    This key filter returns always false (including unknown keys).
\end{key}

\begin{key}{/pgf/key filters/defined}
    This key filter returns false if the current key is unknown, which avoids
    calling the unknown handlers.
\end{key}


\subsubsection{Programmer Interface}
\label{section-key-filter-api}

\begin{plainenvironment}{{pgfkeysinterruptkeyfilter}}
    Temporarily disables key filtering inside the environment. If key filtering
    is not active, this has no effect at all.

    Please note that no \TeX-group is introduced.
\end{plainenvironment}

\begin{command}{\pgfkeyssavekeyfilterstateto\marg{macro}}
    Creates \marg{macro} which contains commands to re-activate the current key
    filter and key filter handler. It can be used to temporarily switch the key
    filter.
\end{command}

\begin{command}{\pgfkeysinstallkeyfilter\marg{full key}\marg{optional arguments}}
    The command
    |\pgfkeysinstallkeyfilter{|\meta{full key}|}{|\meta{optional arguments}|}|
    has the same effect as
    |\pgfkeys{|\meta{full key}|/.install key filter={|\meta{optional arguments}|}}|.
\end{command}

\begin{command}{\pgfkeysinstallkeyfilterhandler\marg{full key}\marg{optional arguments}}
    The command
    |\pgfkeysinstallkeyfilterhandler{|\meta{full key}|}{|\meta{optional arguments}|}|
    has the same effect as
    |\pgfkeys{|\meta{full key}|/.install key filter handler={|\meta{optional arguments}|}}|.
\end{command}

\begin{command}{\pgfkeysactivatefamily\marg{family name}}
    Equivalent to |\pgfkeys{|\meta{family name}|/.activate family}|.
\end{command}

\begin{command}{\pgfkeysdeactivatefamily\marg{family name}}
    Equivalent to |\pgfkeys{|\meta{family name}|/.deactivate family}|.
\end{command}

\begin{command}{\pgfkeysactivatefamilies\marg{family list}\marg{deactivate macro name}}
    Activates each family in \meta{family list} and creates a macro
    \meta{deactivate macro name} which deactivates each family in \meta{family
    list}.
    %
\begin{codeexample}[code only,preamble={\usepgfkeyslibrary{filtered}}]
\pgfkeysactivatefamilies{/family 1,/family 2,/family 3}{\deactivatename}
\pgfkeysfiltered{foo,bar}
\deactivatename
\end{codeexample}
    %
\end{command}

\begin{command}{\pgfkeysiffamilydefined\marg{family}\marg{true case}\marg{false case}}%
    Checks whether the full key \meta{family} is a family and executes either
    \meta{true case} or \meta{false case}.
\end{command}

\begin{command}{\pgfkeysisfamilyactive\marg{family}}
    Sets the \TeX-boolean |\ifpgfkeysfiltercontinue| to whether \meta{family}
    is active or not.
\end{command}

\begin{command}{\pgfkeysgetfamily\marg{full key}\marg{resultmacro}}
    Returns the family associated to a \meta{full key} into macro
    \meta{resultmacro}.
\end{command}

\begin{command}{\pgfkeyssetfamily\marg{full key}\marg{family}}
    The command |\pgfkeyssetfamily|\marg{full key}\marg{family} has the same
    effect as |\pgfkeys{|\meta{full key}|/.belongs to family=|\marg{family}|}|.
\end{command}


\subsubsection{Defining Own Filters Or Filter Handlers}
\label{section-key-writing-filters}

During |\pgfkeysfiltered|, the key filter code will be invoked. At this time,
the full key path including key name is available as |\pgfkeyscurrentkey|, the
key name before default paths have been considered as |\pgfkeyscurrentkeyRAW|
and the values as |\pgfkeyscurrentvalue|.

Furthermore, the macro |\pgfkeyscasenumber| contains the current key's type as
an integer:
%
\begin{itemize}
    \item[\meta{1}] The key is a command key (i.e.\ |.../.@cmd| exists).
    \item[\meta{2}] The key contains its value directly.
    \item[\meta{3}] The key is handled (for example it is |.code| or |.cd|).

        In this case, the macros |\pgfkeyscurrentname| and
        |\pgfkeyscurrentpath| are set to the handlers name and path,
        respectively. Invoke |\pgfkeyssplitpath{}| to extract these values for
        non-handled keys.
    \item[\meta{0}] The key is unknown.
\end{itemize}
%
Any key filter or key filter handler can access these variables. Key filters
are expected to set the \TeX-boolean |\ifpgfkeysfiltercontinue| to whether the
current key shall be processed or not.

\begin{command}{\pgfkeysevalkeyfilterwith\marg{full key}=\marg{filter arguments}}
    Evaluates a fully qualified key filter \meta{full key} with argument(s)
    \meta{filter arguments}.
    %
\begin{codeexample}[code only,preamble={\usepgfkeyslibrary{filtered}}]
\pgfkeysevalkeyfilterwith{/pgf/key filters/equals=/tikz}
\end{codeexample}
    %
\end{command}

\endgroup
